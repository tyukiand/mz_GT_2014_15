\documentclass{scrartcl}

\usepackage{amsmath}	  % required for math in general
\usepackage{amsthm}     % environments for theorems, qed's etc
                        % (loaded after amsmath)
\usepackage{amssymb}	  % doublestroke symbols, other mathematical symbols
\usepackage{dsfont}     % required for double-stroke 1 as characteristic function
\usepackage{array}	    % control of matrices and tables
\usepackage[pdftex]{graphicx}   % images

\usepackage{enumitem}   % more fine-grained control over enumerations
\setdescription{leftmargin=\parindent,labelindent=\parindent}

\usepackage{listings} % code listings
\lstset{basicstyle=\ttfamily\scriptsize}

% \input{diagrams.sty} (no category theory this time)

\usepackage{helvet}   % use (much fresher looking) helvetica for everything
\renewcommand{\familydefault}{\sfdefault}

\usepackage[weather]{ifsym}      % \Lightning symbol
% \usepackage{mathabx}             % \Asterisk causes some conflicts

% forcing the fucking floats to stop fucking floating like a fucking piece of
% shit in an ocean of fucking shit
\renewcommand{\topfraction}{.85}
\renewcommand{\bottomfraction}{.7}
\renewcommand{\textfraction}{.15}
\renewcommand{\floatpagefraction}{.66}
\renewcommand{\dbltopfraction}{.66}

% making all references into hyperlinks
\usepackage[dvipsnames]{xcolor}
\usepackage{hyperref}

\hypersetup{colorlinks=true,linkcolor=MidnightBlue,pdfborderstyle={/W 0}}

\usepackage{anyfontsize}
\usepackage{datetime}

% stuff used by Felix
\usepackage[utf8]{inputenc}
\usepackage[T1]{fontenc}
%\usepackage{lmodern} (moar fonts? does one really need it?)
\usepackage[ngerman]{babel}
\usepackage{tabularx}
\usepackage{multirow}
\usepackage{amsfonts}
\usepackage{tabto}
\usepackage{mathtools}
\TabPositions{0.1in, 0.4in, 0.6in, 0.8in, 1.0in, 1.2in, 3.4in}

\usepackage{breqn} % breaking equations 

\input{./definitions}
\newcommand{\exercise}[2]{\vspace{1em}\noindent{\bf Exercise #1 (#2)}}
\newcommand{\subexercise}[1]{\vspace{0.8em}\noindent{\bf #1)}}
\renewcommand{\proof}{\vspace{0.8em}\noindent{\bf Proof: }}

\begin{document}
\noindent{\footnotesize Game Theory 2014/15, Exercise 11} 
\hfill 
{\footnotesize \input{./currentDate.txt}}
\newline
{\footnotesize \input{../../NAMES.txt}}

\noindent\hrulefill

\exercise{11.1}{Iterated games with decaying payoffs}
We consider the iterated prisoner's dilemma with a decay coefficient 
$\lambda\in(0, 1)$. The ``gross'' payoffs in each round are given by the 
following table:
\[
  \mat{ccc}{
                       & \textrm{Defect} & \textrm{Cooperate} \\
    \textrm{Defect}    &  (1, 1)         & (4, 0) \\
    \textrm{Cooperate} &  (0, 4)         & (3, 3)
  }
\]
The total payoff of the player $p$ is given by:
\[
  U_p(\sigma_1, \sigma_2) = 
    \E\sPar{\sum_{t=0}^\infty\lambda^t u_p(\sigma_1^t, \sigma_2^t)}.
\]
Notice that we start in the round $t = 0$.

\subexercise{a} We want to investigate for which $\lambda$'s 
the \emph{Grim trigger} strategy is a Nash equilibrium.

\vspace{0.5em}
\noindent{\bf Claim: } $(GT,GT)$ is a Nash equilibrium if and only if 
$\lambda \geq 1/3$.

\vspace{0.5em}
\noindent{\bf Proof: } Suppose $\lambda \geq 1/3$. Let $\sigma$ be any 
other strategy. 
Define the \emph{stopping time}\footnote{See next footnote} $\tau$ as follows:
\[
  \tau := \inf\setPredicate{t \in \Natural_0}{\sigma^t = D},
\]
that is, it is the index of the round when $\sigma$ plays $D$ 
for the first time.
It holds:
\begin{align*}
  \begin{array}{>{\displaystyle}rc>{\displaystyle}l}
    \sum_{t=0}^\infty \lambda^t u_1(\sigma^t, TT^t)
     & = & \sum_{t = 0}^{\tau - 1} \lambda^t u_1(C, C) + 
           \lambda^\tau u_1(D, C) + 
           \sum_{t = \tau + 1}^\infty \lambda^t u_1(\sigma^t, D) \\
     & \leq &
       \sum_{t = 0}^{\tau - 1} \lambda^t \cdot 3 + 4 \lambda^\tau + 
       \sum_{t = \tau + 1}^\infty \lambda^t \cdot 1 \\
     & = & 3\frac{1 - \lambda^\tau}{1 - \lambda} + 4\lambda^\tau +
           \frac{\lambda^{\tau + 1}}{1 - \lambda}.
  \end{array}
\end{align*}
Compare this expression to the payoff for playing $GT$:
\[
  U_1(GT, GT) = \frac{3}{1-\lambda}.
\]
It holds:
\begin{align*}
  \begin{array}{crcl}
     & 3\frac{1 - \lambda^\tau}{1 - \lambda} + 4\lambda^\tau +
           \frac{\lambda^{\tau + 1}}{1 - \lambda}
    & \leq & \frac{3}{1-\lambda} \\
     \Leftrightarrow & 
    3 + \lambda^\tau - 3\lambda^{\tau + 1} & \leq  & 3 \\
     \Leftrightarrow &
    \lambda^{\tau}(1 - 3 \lambda) & \leq & 0 \\
     \Leftrightarrow & 
      \lambda & \geq & \frac{1}{3},
  \end{array}
\end{align*}
therefore by monotonicity of the expected value:
\[
  U_1(\sigma, GT) = \E\sPar{\sum_{t=0}^\infty \lambda^t u_1(\sigma^t, GT^t)}
    \leq 
    \E\sPar{\sum_{t=0}^\infty \lambda^t u_1(GT^t, GT^t)}
    = U_1(GT, GT),
\]
hence $\lambda \geq 1/3$ implies that there are no advantageous deviations from
$GT$, i.e. $(GT, GT)$ is a Nash equilibrium.

To show the opposite implication, assume that $\lambda < 1/3$.
Consider the strategy \emph{Always-Defect}. It holds:
\[
  U_1(D, GT) = 4 + \frac{\lambda}{1-\lambda} = \frac{4-3\lambda}{1-\lambda}.
\]
Since we assumed $\lambda < 1/3$ it holds:
\[
  U_1(D, GT) = \frac{4 - 3\lambda}{1-\lambda} > \frac{3}{1-\lambda} 
             = U_1(GT, GT),
\]
therefore we have found an advantageous deviation from $GT$. This means that
$(GT, GT)$ is not a Nash equilibrium. 

Both implications together give the claimed equivalence. \hfill \qed

\subexercise{b} Now we wish to understand for what $\lambda$'s the 
Tit-for-Tat strategy is a Nash-equilibrium.
Before giving a formal proof, we give an ansatz: an informal 
heuristic that shows how one could come up with the critical value
$\lambda_{crit} = 1/3$.

  One simple ``exploit'' of the Tit-for-Tat strategy the
player $1$ could attempt is the following.
Player $1$ can try to deviate from $C$ to $D$, 
then continue to play $D$ for $(k-1)$ rounds, 
then change back to $C$. The game then looks like this:

\vspace{0.5em}
% TAG CENTER CENTERING TABLE TABULAR (makebox solution)
\makebox[\linewidth]{
  \centering
  \begin{tabular}{c|ccccccc}
    $\sigma_1$ & D & D & D & D & \dots & D & C \\
    $TT$ &       C & D & D & D & \dots & D & D \\
    $u_1$ & 4 & 1 & 1 & 1 & \dots & 1 & 0
  \end{tabular}
}

\vspace{0.5em}
The ansatz is that this maneuvre by the player $1$ should give 
him no advantage for the critical $\lambda_{crit}$, that is, the outcome
for this block should be the same as if he would just play $C$
all the time, and get a gross payoff of $3$ per round:
\[
  4 + \lambda + \lambda^2 + \dots + \lambda^{k-1} + 0 
  \overset{!}{=}
  3(1 + \lambda + \lambda^2 + \dots + \lambda^{k-1} + \lambda^{k})
\]
Simplifying both expressions using the formula for geometric series 
we obtain:
\[
  3 + \frac{1 - \lambda^k}{1 - \lambda} 
  \overset{!}{=}
  3\frac{1 - \lambda^{k + 1}}{1 - \lambda}
\]
After some shuffling of the summands we get:
\[
  (1-3\lambda) \overset{!}{=} (1-3\lambda)\lambda^k.
\]
The only way this can hold for all $k$ is that 
$\lambda_{crit}\equiv\lambda = \frac{1}{3}$.

Now we have some intuition and a plausible claim. Before we give a formal proof, let's
take a quick look at some potential pitfalls. Notice that:
\begin{itemize}
  \item Coming up with some exploit that is better than $TT$ for some $\lambda$
    gives only a lower bound for the critical lambda, it does not show that
    there are no advantageous deviations for larger $\lambda$'s;
  \item If one tries to prove something considering only hard-coded 
    non-randomized strategies, one needs some argument that guarantees that 
    there are no randomized strategies that are better;
  \item If one tries to prove something considering only ``the previous step''
    or some finite number of ``previous steps'' (kind of Markovian property), 
    one needs some extra argument that guarantees that there are no better strategies
    with built-in long-term memory.
\end{itemize}

\vspace{0.5em}
\noindent{\bf Claim:} The strategy $(TT, TT)$ is a Nash equilibrium if and only
if $\lambda \geq \frac{1}{3}$ .

\vspace{0.5em}
\noindent{\bf Proof:} First, let's introduce a notation for the payoff of a 
block that starts at time $0$ and goes until time $\zeta$, with player $1$
starting to play $D$ at time $\delta$ and switching back to $C$ at time $\zeta$.
Here a sketch of what we are trying to compute:
\[
  \mat{c|cccccccccccc}{
    t        & 0 & 1 &   &   & \dots &   & \delta &   &   & \dots &   & \zeta \\
    \sigma_1 & C & C & C & C & \dots & C & D      & D & D & \dots & D & C \\
    TT       & C & C & C & C & \dots & C & C      & D & D & \dots & D & D \\
    u_1      & 3 & 3 & 3 & 3 & \dots & 3 & 4      & 1 & 1 & \dots & 1 & 0
  }
\]
Here is the definition, combined with an immediate simplification:
\begin{align*}
  block_\lambda(\delta, \zeta) 
    &:= 
    \sum_{t = 0}^{\delta - 1}\lambda^t u_1(C, C) + \lambda^\delta u_1(D, C) +
    \sum_{t = \delta + 1}^{\zeta - 1}\lambda^t u_1(D, D) +
    \lambda^\zeta u_1(C, D) \\
    &=
    \frac{1-\lambda^\delta}{1-\lambda}\cdot 3 +
    \frac{\lambda^\delta - \lambda^{\delta + 1}}{1 - \lambda} \cdot 4 +
    \frac{\lambda^{\delta + 1} - \lambda^{\zeta}}{1 - \lambda} \cdot 1 + 0 \\ 
    &= 
    \frac{1}{1-\lambda}\rPar{
      3 + \lambda^\delta - 3\lambda^{\delta + 1} - \lambda^\zeta
    }.
\end{align*}
We want to compare this block to the outcome of just playing $C$, that is,
we want to show that for all $\delta$ and $\zeta$ it holds:
\[
  block_\lambda(\delta, \zeta) \leq \sum_{t=0}^{\zeta}\lambda^t\cdot 3 
    = \frac{3 - 3\lambda^{\zeta + 1}}{1 - \lambda},
\]
if $\lambda \geq 1/3$.
For this, consider the following equivalences:
% TAG LATEX ALIGN ALIGN ARRAY EQUIVALENCES (bunch of <=> with centered \leq)
\begin{align*}
  \begin{array}{crl}
                  & \frac{1}{3} &\leq \lambda \\
  \Leftrightarrow & 1           &\leq 3\lambda \\
  \Leftrightarrow & \lambda^\delta(1-\lambda^{\zeta-\delta}) &\leq 
                    3\lambda^{\delta+1}(1-\lambda^{\zeta-\delta}) \\
  \Leftrightarrow & \lambda^\delta - 3\lambda^{\delta+1} - \lambda^\zeta &\leq
                    -3\lambda^{\zeta + 1} \\
  \Leftrightarrow & \frac{
      3 + \lambda^\delta - 3\lambda^{\delta + 1} - \lambda^\zeta
    }{1-\lambda} & \leq \frac{3-3\lambda^{\delta+1}}{1-\lambda} \\
  \Leftrightarrow & block_\lambda(\delta, \zeta) &\leq 
    \sum_{t=0}^{\zeta}\lambda^t\cdot 3.
  \end{array}
\end{align*}
Now let $\sigma$ be an arbitrary strategy of player $1$. We want to show that
for $\lambda \geq 1/3$, the expected payoff is less or equal to the payoff for
$TT$. For this, we will ``cut'' the game into blocks like the one above, and 
treat each block separately. For convenience, set $\sigma^{-1} := C$.
Define a sequence of \emph{stopping times}\footnote{
Although ``\emph{stopping time}'' has a precise
meaning, we do not want to go too much 
into details here. Just think of $\delta_i$ and $\zeta_i$ 
(for $i \in \Natural_0$) as of 
$\Natural_0 \cup \set{\infty}$-valued random variables that tell us 
when player $1$ switches from $C$ to $D$ resp. from $D$ to $C$. 
These quantities have to be random, because $\sigma$ can be randomized.
} $\delta_i, \zeta_i$ as follows: 
\begin{align*}
  \zeta_{0}  &:= -1  \\
  \delta_{i} &:= \inf\setPredicate{t > \zeta_{i-1}}{\sigma^t = D} \\
  \zeta_{i}  &:= \inf\setPredicate{t > \delta_{i}}{\sigma^t = C}
\end{align*}
The $\zeta_{0}$ together with $\sigma^{-1} := C$ are set this way just for 
convenience, so that we do not need two cases in the definition of $\delta_i$.
Notice that $\inf$ includes the possibility that $\delta_i$ or $\zeta_i$ remain
infinite, that is, at some point the player stops switching between $C$ and $D$.
To keep the notation simple yet consistent, we henceforth treat expressions
like $\sum_{t = \infty}^\infty f(t)$ as empty sums with value $0$. Also notice
that the definition of $block_\lambda$ is compatible with this convention and
can accept $\infty$ as arguments.

The stopping times now let us express the payoff as a sum (or series) of 
payoffs for simple blocks, contribution of each block can be bounded using the
previous estimations:
% TAGS: LATEX ARRAY ALIGN EQUATIONS ALIGNMENT OVERSET (\displaystyle in array)
\begin{align*}
  \begin{array}{>{\displaystyle}rc>{\displaystyle}l} 
    \sum_{t=0}^\infty \lambda^t u_1(\sigma^t, TT^t)
    & = & 
    \sum_{i = 1}^{\infty}\rPar{
    \sum_{t=\zeta_{i-1}+1}^{\delta_i - 1}\lambda^t u_1(\sigma^t,\sigma^{t-1}) +
    \lambda^{\delta_i}u_1(\sigma^{\delta_i}, \sigma^{\delta_i - 1}) + \right. \\
    & &\hspace{3em}\left.
    \sum_{t=\delta_{i}+1}^{\zeta_i - 1} \lambda^t u_1(\sigma^t, \sigma^{t-1}) +
    \lambda^{\zeta_i}u_1(\sigma^{\zeta_i}, \sigma^{\zeta_i - 1})} \\
  & \overset{\textrm{def }\delta_i, \zeta_i}{=} &
    \sum_{i = 1}^{\infty}\rPar{
    \sum_{t=\zeta_{i-1}+1}^{\delta_i - 1}\lambda^t u_1(C, C) +
    \lambda^{\delta_i}u_1(D, C) + \right. \\
     & &\hspace{3em}\left.
    \sum_{t=\delta_{i}+1}^{\zeta_i - 1} \lambda^t u_1(D, D) +
    \lambda^{\zeta_i}u_1(C, D)} \\
  & \overset{\textrm{def }block}{=} &
    \sum_{i = 1}^{\infty}\lambda^{\zeta_{i-1} + 1} 
    block_\lambda\rPar{
      \delta_i - (\zeta_{i-1} + 1), \zeta_i - (\zeta_{i-1} + 1)
    } \\
  & \leq &
    \sum_{i = 1}^{\infty}\lambda^{\zeta_{i-1} + 1} 
    \sum_{t = 0}^{\zeta_i - (\zeta_{i-1} + 1)}\lambda^t \cdot 3 \\
  & = &
    \sum_{i = 1}^{\infty}
    \sum_{t = \zeta_{i-1} + 1}^{\zeta_i}\lambda^t \cdot 3 
  \, = \,
    \sum_{t = 0}^{\infty}\lambda^t \cdot 3 \\
  & = &
    \sum_{t = 0}^{\infty}\lambda^t u_1(TT^t, TT^t).
  \end{array}
\end{align*}
Now the payoff for $\sigma$ is bounded by the payoff for $TT$, by monotonicity
of the expected value it holds:
\[
  U_1(\sigma, TT) 
    = \E\sPar{\sum_{t=0}^\infty \lambda^t u_1(\sigma^t, TT^t)}
    \leq \E\sPar{\sum_{t=0}^\infty \lambda^t u_1(TT^t, TT^t)}
    = U_1(TT, TT).
\]
This shows: for all $\lambda \geq 1/3$ there are no advantageous deviations from
the strategy $TT$, so that $(TT, TT)$ is a Nash equilibrium.
Now we have to show the opposite direction by finding an advantageous deviation
in the case that $\lambda<1/3$. For this, consider the following strategy:
\[
  \sigma^t := \begin{cases}
    D &\textrm{for even }t \\
    C &\textrm{for odd }t.
  \end{cases}
\]
It holds:
\begin{align*}
  \sum_{t=0}^\infty \lambda^t u_1(\sigma^t, TT^t)
    =& \sum_{i = 0}^\infty \lambda^{2i} \rPar{u_1(D, C) + \lambda u_1(C, D)}
    = \sum_{i = 0}^\infty \lambda^{2i} \rPar{4 + \lambda\cdot 0} \\
    >& \sum_{i = 0}^\infty \lambda^{2i} \rPar{3 + \lambda\cdot 3} 
    = \sum_{i = 0}^\infty \lambda^{2i} \rPar{u_1(C, C) + \lambda u_1(C, C)} \\
    =& \sum_{t = 0}^\infty \lambda^t u_1(TT^t, TT^t),
\end{align*}
that is, if $\lambda < 1/3$, player $1$ can deviate from $TT$ and gain an
edge over player $2$. In particular, $(TT, TT)$ is not a Nash equilibrium.

Both directions together show the equivalence of statements 
``$\lambda \geq 1/3$'' and ``$(TT, TT)$ is a Nash equilibrium''. \hfill \qed


\exercise{11.2}{Auktionen}

\subexercise{a}
Wir versuchen, den erwarteten Nutzen für jeden Spieler zu maximieren. Wir nehmen dabei an,
dass $b_i \leq 2$, da es nicht sinnvoll ist, ein Gebot abzugeben, welches größer ist als der
mögliche Wert des Gegenstandes für einen Spieler. \\

Wir zeigen, dass folgende Strategie für jeden Spieler ein symmetrisches Gleichgewicht ist: \\

\[
b_i(v_i) = \frac{v_i}{2}, \quad i = 1, 2, 3
\]

Wir betrachten ohne Einschränkung den Nutzen für Spieler I und bezeichnen mit $b^*$ das
Gebot, welches abgesehen vom Gebot von Spieler I, das höchste Gebot sei (unter der
Annahme, dass diese auch diese Strategie spielen - ihre Gebote seien daher $b^*_2$ und
$b^*_3$): \\

\[
b^* = \frac{v^*}{2}= max \{b^*_2, b^*_3\} = max \{\frac{v_2}{2}, \frac{v_3}{2}\}
\]

Damit ist der Nutzen für Spieler 1: \\

\[
u_1(b_1, b^*, v_1) = u_1(b_1, \frac{v^*}{2}, v_1)
\]
\[
= P(b_1 > \frac{v^*}{2}) \cdot (v_1 - b_1)
\]
\[
= P(2b_1 > v^*) \cdot (v_1 - b_1)
\]
\[
= min\{2b_1,1\} \cdot (v_1 - b_1)
\]

Diese Funktion ist quadratisch auf dem Intervall $b_1 \in [0, \frac{1}{2}]$ (erreicht
das Maximum bei $b_1 = \frac{v_1}{2}$), und linear mit negativer Steigung, wenn
$b_1 > \frac{1}{2}$. \\

Fazit: Wenn die beiden anderen Spieler die Strategie $b_i = \frac{v_i}{2}$ spielen,
so ist für Spieler I $b_1 = \frac{v_1}{2}$ die beste Antwort. Damit ist $b^* =
b_i(v_i) = \frac{v_i}{2}$ ein symmetrisches Gleichgewicht.

\subexercise{b} [omitted]
\end{document}