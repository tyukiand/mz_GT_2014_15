\documentclass{scrartcl}

\usepackage{amsmath}	  % required for math in general
\usepackage{amsthm}     % environments for theorems, qed's etc
                        % (loaded after amsmath)
\usepackage{amssymb}	  % doublestroke symbols, other mathematical symbols
\usepackage{dsfont}     % required for double-stroke 1 as characteristic function
\usepackage{array}	    % control of matrices and tables
\usepackage[pdftex]{graphicx}   % images

\usepackage{enumitem}   % more fine-grained control over enumerations
\setdescription{leftmargin=\parindent,labelindent=\parindent}

\usepackage{listings} % code listings
\lstset{basicstyle=\ttfamily\scriptsize}

% \input{diagrams.sty} (no category theory this time)

\usepackage{helvet}   % use (much fresher looking) helvetica for everything
\renewcommand{\familydefault}{\sfdefault}

\usepackage[weather]{ifsym}      % \Lightning symbol
% \usepackage{mathabx}             % \Asterisk causes some conflicts

% forcing the fucking floats to stop fucking floating like a fucking piece of
% shit in an ocean of fucking shit
\renewcommand{\topfraction}{.85}
\renewcommand{\bottomfraction}{.7}
\renewcommand{\textfraction}{.15}
\renewcommand{\floatpagefraction}{.66}
\renewcommand{\dbltopfraction}{.66}

% making all references into hyperlinks
\usepackage[dvipsnames]{xcolor}
\usepackage{hyperref}

\hypersetup{colorlinks=true,linkcolor=MidnightBlue,pdfborderstyle={/W 0}}

\usepackage{anyfontsize}
\usepackage{datetime}

% stuff used by Felix
\usepackage[utf8]{inputenc}
\usepackage[T1]{fontenc}
%\usepackage{lmodern} (moar fonts? does one really need it?)
\usepackage[ngerman]{babel}
\usepackage{tabularx}
\usepackage{multirow}
\usepackage{amsfonts}
\usepackage{tabto}
\usepackage{mathtools}
\TabPositions{0.1in, 0.4in, 0.6in, 0.8in, 1.0in, 1.2in, 3.4in}

\usepackage{breqn} % breaking equations 

\input{./definitions}
\newcommand{\exercise}[2]{\vspace{1em}\noindent{\bf Exercise #1 (#2)}}
\newcommand{\subexercise}[1]{\vspace{0.8em}\noindent{\bf #1)}}
\renewcommand{\proof}{\vspace{0.8em}\noindent{\bf Proof: }}

\begin{document}
\noindent{\footnotesize Game Theory 2014/15, Exercise 11} 
\hfill 
{\footnotesize \input{./currentDate.txt}}
\newline
{\footnotesize \input{../../NAMES.txt}}

\noindent\hrulefill

\exercise{1.1}{Text with a bunch of funny formulas}

Let $a\in\Real^3$ arbitrary, and let $[a]_\cross$ denote the matrix that
corresponds to the cross product with $a$ from the left:
\[
  [a]_\cross := \mat{ccc}{
    0    & -a_3 & a_1 \\
    a_3  & 0    & -a_2 \\
    -a_1 & a_2  & 0
  }.
\]
It then holds: 
\[
  \exp([a]_\cross) = R_{\frac{a}{\norm{a}}}{\norm{a}},
\]
where $R_{z,\phi}$ is the rotation matrix
around the axis $z$ by the angle $\phi$.

\proof Observe that all powers of $[a]_\cross$ can be 
expressed in terms of $[a]_\cross$ and $[a]_\cross^2$, 
more precisely, it holds (by the Grassmann-Identity):
\begin{align*}
  [a]_\cross^2 &= aa^T - \norm{a}^2I  \\
  [a]_\cross^3 &= -\norm{a}^2[a]_\cross.
\end{align*}
Iteration gives us explicit formulas for all nonnegative powers:
\[
  [a]_\cross^k = \begin{cases}
    I & \textrm{ for } k = 0 \\
    \rPar{(-1)\norm{a}^2}^\frac{k-1}{2}[a]_\cross & \textrm{ for } k \textrm{ odd} \\
    \rPar{-\norm{a}^2}^{\frac{k}{2} - 1}[a]_\cross^2 & 
      \textrm{ for } k \textrm{ even and positive }
  \end{cases}.
\]
Now we start with the series for $\exp$, sort all the terms by 
their matrix-component, and compare the resulting series with $\sin$ and $\cos$:
\begin{align*}
  \exp([a]_\cross) 
    &= \sum_{k=0}^\infty \frac{[a]_\cross^k}{k!}\\
    &= I + \rPar{
         \sum_{k = 1, k \textrm{ odd}}^\infty
         \frac{(-1)^{\frac{k-1}{2}}\norm{a}^{k}}{k!}
       }\sPar{\frac{a}{\norm{a}}}_\cross
       + 
      \rPar{
        \sum_{k = 2, k \textrm{ even}}^\infty
        \frac{-(-1)^\frac{k}{2}\norm{a}^k}{k!}
      }\sPar{\frac{a}{\norm{a}}}_\cross^2 \\
     &= I + \sin(\norm{a})\sPar{\frac{a}{\norm{a}}}_\cross + 
        (1 - \cos(\norm{a}))\sPar{\frac{a}{\norm{a}}}_\cross^2.
\end{align*}
We have already shown that the last expression is exactly the 
rotation matrix $R_{\frac{a}{\norm{a}}}{\norm{a}}$, 
this concludes the proof.  \hfill\qed

\exercise{1.2}{A funny image}
Here you can see an example of an image.

% \begin{figure}[htbp]
%   \centering\includegraphics[width=0.5\linewidth]{images/exampleImage.png}
% \end{figure}
And here is some more text:
\begin{align*}
  [a]_\cross^2 &= aa^T - \norm{a}^2I  \\
  [a]_\cross^3 &= -\norm{a}^2[a]_\cross.
\end{align*}
Iteration gives us similar expressions for all positive powers:
\[
  [a]_\cross^k = \begin{cases}
    I & \textrm{ for } k = 0 \\
    \rPar{(-1)\norm{a}^2}^\frac{k-1}{2}[a]_\cross & \textrm{ for } k \textrm{ odd} \\
    \rPar{-\norm{a}^2}^{\frac{k}{2} - 1}[a]_\cross^2 & 
      \textrm{ for } k \textrm{ even and positive }
  \end{cases}.
\]
Now we start with the series for $\exp$, sort all the terms by 
their matrix-component, and compare the resulting series with $\sin$ and $\cos$:
\begin{align*}
  \exp([a]_\cross) 
    &= \sum_{k=0}^\infty \frac{[a]_\cross^k}{k!}\\
    &= I + \rPar{
         \sum_{k = 1, k \textrm{ odd}}^\infty
         \frac{(-1)^{\frac{k-1}{2}}\norm{a}^{k}}{k!}
       }\sPar{\frac{a}{\norm{a}}}_\cross
       + 
      \rPar{
        \sum_{k = 2, k \textrm{ even}}^\infty
        \frac{-(-1)^\frac{k}{2}\norm{a}^k}{k!}
      }\sPar{\frac{a}{\norm{a}}}_\cross^2 \\
     &= I + \sin(\norm{a})\sPar{\frac{a}{\norm{a}}}_\cross + 
        (1 - \cos(\norm{a}))\sPar{\frac{a}{\norm{a}}}_\cross^2.
\end{align*}
Blah blah blah...

\exercise{1.3}{Code-examples}
Here is how one can include formatted code:
\begin{lstlisting}
  ...
  void main(void) {
    cout << "blah";
  }
\end{lstlisting}
You might also find \verb|\lstinputlisting[firstline=n,lastline=m]| useful.
\end{document}
