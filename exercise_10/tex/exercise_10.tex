\documentclass{scrartcl}

\usepackage{amsmath}	  % required for math in general
\usepackage{amsthm}     % environments for theorems, qed's etc
                        % (loaded after amsmath)
\usepackage{amssymb}	  % doublestroke symbols, other mathematical symbols
\usepackage{dsfont}     % required for double-stroke 1 as characteristic function
\usepackage{array}	    % control of matrices and tables
\usepackage[pdftex]{graphicx}   % images

\usepackage{enumitem}   % more fine-grained control over enumerations
\setdescription{leftmargin=\parindent,labelindent=\parindent}

\usepackage{listings} % code listings
\lstset{basicstyle=\ttfamily\scriptsize}

% \input{diagrams.sty} (no category theory this time)

\usepackage{helvet}   % use (much fresher looking) helvetica for everything
\renewcommand{\familydefault}{\sfdefault}

\usepackage[weather]{ifsym}      % \Lightning symbol
% \usepackage{mathabx}             % \Asterisk causes some conflicts

% forcing the fucking floats to stop fucking floating like a fucking piece of
% shit in an ocean of fucking shit
\renewcommand{\topfraction}{.85}
\renewcommand{\bottomfraction}{.7}
\renewcommand{\textfraction}{.15}
\renewcommand{\floatpagefraction}{.66}
\renewcommand{\dbltopfraction}{.66}

% making all references into hyperlinks
\usepackage[dvipsnames]{xcolor}
\usepackage{hyperref}

\hypersetup{colorlinks=true,linkcolor=MidnightBlue,pdfborderstyle={/W 0}}

\usepackage{anyfontsize}
\usepackage{datetime}

% stuff used by Felix
\usepackage[utf8]{inputenc}
\usepackage[T1]{fontenc}
%\usepackage{lmodern} (moar fonts? does one really need it?)
\usepackage[ngerman]{babel}
\usepackage{tabularx}
\usepackage{multirow}
\usepackage{amsfonts}
\usepackage{tabto}
\usepackage{mathtools}
\TabPositions{0.1in, 0.4in, 0.6in, 0.8in, 1.0in, 1.2in, 3.4in}

\usepackage{breqn} % breaking equations 

\input{./definitions}
\newcommand{\exercise}[2]{\vspace{1em}\noindent{\bf Exercise #1 (#2)}}
\newcommand{\subexercise}[1]{\vspace{0.8em}\noindent{\bf #1)}}
\renewcommand{\proof}{\vspace{0.8em}\noindent{\bf Proof: }}

\begin{document}
\noindent{\footnotesize Game Theory 2014/15, Exercise 10} 
\hfill 
{\footnotesize \input{./currentDate.txt}}
\newline
{\footnotesize \input{../../NAMES.txt}}

\noindent\hrulefill

\exercise{10.1}{Always-Defect and Tit-for-Tat}

\subexercise{a}
Die Strategie $C$ wird von $D$ strikt dominiert. Das heißt, es ist für jeden der Spieler (egal
welche gemischte Strategie gerade von ihm gespielt wird) rational, mehr $D$ zu spielen als er es gerade
tut (falls er nicht eh schon die reine Strategie $D$ spielt). \\

Es kann also außer \textsl{always defect} kein anderes Nash-Gleichgewicht geben, da einer der
Spieler seiner Nutzen immer erhöhen kann. Soweit für das einfache Spiel. \\

Im endlich oft iterierten Spiel nehmen wir nun an, es gebe ein Nash-Gleichgewicht, welches nicht
\textsl{always defect} sei. Da die Spieler wissen, welches die letzte Iteration ist, können sie
in dieser Iteration $D$ spielen, da sie nicht vom anderen dafür "`bestraft"' werden können. Daher
kann diese Strategie kein Nash-Gleichgewicht sein, da es sich für jeden der Spieler lohnnt,
abzuweichen. \\

\subexercise{b}
Da $AD^t_I = AD^t_{II} = D$ für alle $t$, folgt, dass:
\[
\lim_{T \to \infty} \frac{1}{T} \sum_{t=1}^{T} U(AD^t_I, AD^t_{II}) = \lim_{T \to \infty}
\frac{1}{T} \sum_{t=1}^{T} U(D, D) = \lim_{T \to \infty} \frac{1}{T} \sum_{t=1}^{T} 1
\]
\[
= \lim_{T \to \infty} \frac{1}{T} \cdot T = \lim_{T \to \infty} 1 = 1
\]

Weicht ein Spieler ab, so erhält er einen Nutzen von 0, der Gegner einen Nutzen von 4 in dieser
Iteration. Er kann nun weiter $C$ spielen und 0 erhalten, oder zurückkehren zu $D$ und wieder 1
bekommen, aber nicht mehr als 1 pro Durchgang. 

\subexercise{c}
Im Gegensatz zu a) wäre im unendlich oft iterierten Spiel \textsl{always cooperate} ein Nash-
Gleichgewicht (Darauf läuft es hinaus, wenn beide Spieler \textsl{tit for tat} spielen und keiner
von $C$ abweicht)$^1$. Das Abweichen und der einmalige Gewinn von 4 zahlen sich über lange Sicht
nicht aus:

I weicht von $C$ ab und spielt $D$, während II noch $C$ spielt. In der nächsten Runde spielt
nun I $C$ (was II zuvor gespielt hat) und II spielt $D$ (was I zuvor gespielt hat). Dies
wiederholt sich offensichtlich von diesem Zeitpunkt an. Damit hat mal Spieler I einen Nutzen von
4, II einen Nutzen von 0, mal umgekehrt. Im Schnitt kommt jeder Spieler also auf einen Nutzen
von 2. Im unendlich oft wiederholten Spiel wäre der durchschnittliche Nutzen für jeden Spieler
also 2. \\

Tit for Tat ohne Abweichung: \\

\[
\lim_{T \to \infty} \frac{1}{T} \sum_{t=1}^{T} U(TFT^t_I, TFT^t_{II}) \overset{1}= \lim_{T \to \infty} 
\frac{1}{T} \sum_{t=1}^{T} U(C, C) = \lim_{T \to \infty} \frac{1}{T} \sum_{t=1}^{T} 3 = 3
\]

Tit for Tat mit Abweichung von Spieler I in Iteration $k$, Nutzen für I: \\

\[
\lim_{T \to \infty} \frac{1}{T} \sum_{t=1}^{k-1} U_I(C, C) + \sum_{t = k+1}^{T}
U_I(TFT'^t_I, TFT'^t_{II}) = \lim_{T \to \infty} \frac{1}{T} ( 3 \cdot k + \sum_{t = k+1}^{T}
2)
\]
\[
= \lim_{T \to \infty} \frac{1}{T} \cdot 3 \cdot k + \lim_{T \to \infty} \frac{1}{T}
\sum_{t = k+1}^{T} 2 = 0 + \lim_{T \to \infty} \frac{1}{T} \cdot (T - k) \cdot 2 = 0 + 2 = 2
\]

Für Spieler II verhält sich der Grenzwert genau so. Ebenso könnte es Spieler II sein, der abweicht.
Da der Erwartungswert (wenn keiner abweicht) höher ist, als wenn einer der Spieler abweicht, ist
\textsl{tit for tat} ein Nash-Gleichgewicht im unendlich oft iterierten Spiel.

\exercise{1.2}{Bayessian equilibria}
Let $G$ be the random type of the game with the distribution
\[
  \P\sPar{G = good} = q, \qquad \P\sPar{G = bad} = (1-q)
\]
for $q\in(0, 1)$. Let $u_p^G(z, (x, y))$ denote the
expected outcome for the player $p$ in the game of type $G$ 
for strategies $z$ of player $I$ and $(x, y)$ of player $II$.

\subexercise{a} 
Let's first compute the expected outcome for player $I$ (averaged over
both game types):
\begin{align*}
  \E\sPar{u_I^G(z, (x,y))} 
    &= q u_I^{good}(z,(x,y)) + (1-q) u_I^{bad}(z, (x, y)) \\
    &= 3qzx + (1-q) zy.
\end{align*}
The partial derivative w.r.t. $z$ is as follows:
\begin{align*}
  \dd{}{z}\E\sPar{u_I^G(z, (x,y))} = 3xq + (1-q)y
\end{align*}
First, we investigate the boundary of $[0,1]$, and search 
Nash equilibria for $z\in\set{0, 1}$.
First, suppose $z=1$. 
The only best response by the player $II$ is $(x, y) = (0,1)$.
The derivative w.r.t. $z$ at $(x, y) = (0, 1)$ is:
\begin{align*}
  \dd{}{z}\E\sPar{u_I^G(z, (0, 1))} = (1 - q) > 0,
\end{align*}
therefore $(z, (x, y)) = (1, (0, 1))$ is a Nash equilibrium
(the only Nash equilibrium for $z = 1$).

Now, suppose $z=0$.
In both games ($good$ or $bad$) it's irrelevant for the player $II$
whether he plays $sell$ or $\neg sell$, all $(x,y) \in [0,1]^2$
are best responses to $z=0$. For the derivative w.r.t. $z$ it holds:
\begin{align*}
  \dd{}{z}\E\sPar{u_I^G(0, (x, y))} \geq 0
\end{align*}
with equality at $(x, y) = (0,0)$. If $(0, (x,y))$ is a Nash equilibrium,
the derivative w.r.t. $z$ has to be $\leq 0$ 
(otherwise $I$ can get better outcome by increasing $z$), 
therefore $(0, (0, 0))$ is the only Nash equilibrium
at $z=0$.

\subexercise{b}
Now we investigate the existence of Nash equilibria for $z\in(0, 1)$.
If $(z, (x, y))$ is a Nash equilibrium, the derivative of the
expected outcome for player $I$ w.r.t. $z$ has to be exactly $0$, that
is:
\[
  0 \overset{!}= \dd{}{z}\E\sPar{u_I^G(z, (x, y))} = 3xq + (1-q)y,
\]
this implies (since all coefficients are nonnegative):
\[
  (x, y) = (0, 0).
\]
Intuitively that means: player $I$ is indifferent only if $II$ does not
sell anything, otherwise $I$ tends to buy the car regardless of it's 
quality.

Now we have to consider $u_{II}$ and check whether $(0, 0)$ is a best
response for some $z\in(0, 1)$. First, consider the $good$ game.
It holds:
\begin{align*}
  u_{II}^{good}(z, (x, y)) &\equiv u_{II}^{good}(z, x) = 3zx + 5 (1-zx) \\
  \dd{}{x} u_{II}^{good}(z, x) &= -2z < 0,
\end{align*}
therefore the gain of $II$ decreases as $x$ increases, so that $x=0$ is the
only best response for all $z\in (0, 1)$.

Similarly, for the $bad$ game it holds:
\begin{align*}
  u_{II}^{bad}(z, (x, y)) &= u_{II}^{bad}(z, y) = 3zy \\
  \dd{}{y} u_{II}^{bad}(z, y) &= 3z > 0,
\end{align*}
therefore the gain of $II$ increases as $y$ increases, so that $y=1$ is
the only best response for all $z\in (0, 1)$.

Now we sum it up: 
if $(z, (x, y))$ is an equilibrium for some $z\in(0,1)$,
then it has to hold: 
\[
  (x,y) \in \set{(0, 0)} \cap \set{(0, 1)} = \emptyset,
\]
that means that there are no equilibria for $z\in (0, 1)$.

Together with a) that means that the set of all equilibria is 
$\set{(0,(0,0)),(1,(0,1))}$. \hfill \qed
\end{document}