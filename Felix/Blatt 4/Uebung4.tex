\documentclass{scrartcl}
 
\usepackage[utf8]{inputenc}
\usepackage[T1]{fontenc}
\usepackage{lmodern}
\usepackage[pdftex]{graphicx}
\usepackage[ngerman]{babel}
\usepackage{amsmath}
\usepackage{amssymb}
\usepackage{tabularx}
\usepackage{multirow}
\usepackage{amsfonts}
\usepackage{tabto}
\usepackage{mathtools}
\TabPositions{0.1in, 0.4in, 0.6in, 0.8in, 1.0in, 1.2in, 3.4in}

\begin{document}
\begin{LARGE}
Spieltheorie - WiSe 2014/15
\end{LARGE}

\begin{Large}
Übungsblatt 4 - Felix Dosch, Julian Felix Rost\\[1.0cm]
\end{Large}

\begin{Large}
Aufgabe 4.1\\[0.0cm]
\end{Large}

a) Wir suchen eine Wahrscheinlichkeitsverteilung 

\[
\sigma_I = \sum_{s_I \in S_I} p(s_I) = p(T) + p(M) + p(B) = 1
\]

, sodass für jede reine Strategie von Spieler II Spieler I den gleichen erwarteten Gewinn
hat. Sei im folgenden $a = p(T), b = p(M), c = p(B)$. Dann ergeben sich folgende erwartete
Gewinne für Spieler I: \\

Spieler II spielt L : $\quad 3a + 2b + 2c$

Spieler II spielt C : $\quad -3a + 6b + 5c$

Spieler II spielt R : $\quad 0a + 4b + 6c$ \\

Aus den Bedingungen der Gleichheit und der zusätzlichen Bedingung $a+b+c = 1$ lässt sich ein
lineares Gleichungssystem mit 3 linear unabhängigen Gleichungen und 3 Variablen aufstellen,
welches eine eindeutige Lösung besitzt:

\[
3a + 2b + 2c = -3a + 6b + 5c
\]
\[
\Leftrightarrow 6a - 4b - 3c = 0 \quad (A)
\]

\[
-3a + 6b + 5c = 4b + 6c
\]
\[
\Leftrightarrow -3a + 2b - c = 0 \quad (B)
\]

\[
a + b + c = 1 \quad (C)
\]

\[
\begin{matrix*}[r]
  (A) & 6a & -&4b & -&3c & = & 0 \\
  (B) & -3a & +&2b & -&c & = & 0 \\
  (C) & a & +&b & +&c & = & 1 \\
\end{matrix*}
\] \[ \cdots \] \[
\begin{matrix*}[r]
  (A) & 25a & & & & & = & 10 \\
  (B) & -2a & +&3b & & & = & 1 \\
  (C) & a & +&b & +&c & = & 1 \\
\end{matrix*}
\] \[ \cdots \] \[
\begin{matrix*}[r]
  (A) & a & & & & & = & \frac{2}{5} \\
  (B) & & & b & & & = & \frac{3}{5} \\
  (C) & & & & & c & = & 0 \\
\end{matrix*}
\]

Es ergibt sich also als gemischte Strategie für Spieler I:
\[
\sigma^*_I = \sum_{s_I \in S_I} p(s_I) \text{ mit } p : S_I \mapsto [0, 1], p(T) = 
\frac{2}{5}, p(M) = \frac{3}{5}, p(B) = 0
\]

Der erwartete Gewinn ist hierbei $3 \cdot \frac{2}{5} + 2 \cdot \frac{3}{5} + 2 \cdot
0 = \frac{12}{5}$. \\

a) Wir suchen eine Wahrscheinlichkeitsverteilung 

\[
\sigma_{II} = \sum_{s_{II} \in S_{II}} p(s_{II}) = p(L) + p(C) + p(R) = 1
\]

, sodass für jede reine Strategie von Spieler I Spieler II den gleichen erwarteten Gewinn
hat. Sei im folgenden $a = p(L), b = p(C), c = p(R)$. Dann ergeben sich folgende erwartete
Gewinne für Spieler II: \\

Spieler I spielt T : $\quad -3a + 3b$

Spieler I spielt M : $\quad -2a - 6b - 4c$

Spieler I spielt B : $\quad -2a - 5b - 6c$ \\

Aus den Bedingungen der Gleichheit und der zusätzlichen Bedingung $a+b+c = 1$ lässt sich ein
lineares Gleichungssystem mit 3 linear unabhängigen Gleichungen und 3 Variablen aufstellen,
welches eine eindeutige Lösung besitzt:

\[
-2a - 6b - 4c = -2a - 5b - 6c
\]
\[
\Leftrightarrow -b + 2c = 0 \quad (A)
\]

\[
-3a + 3b = -2a - 6b - 4c
\]
\[
\Leftrightarrow -a + 9b + 4c = 0 \quad (B)
\]

\[
a + b + c = 1 \quad (C)
\]

\[
\begin{matrix*}[r]
  (A) & & -&b & +&2c & = & 0 \\
  (B) & -a & +&9b & +&4c & = & 0 \\
  (C) & a & +&b & +&c & = & 1 \\
\end{matrix*}
\] \[ \cdots \] \[
\begin{matrix*}[r]
  (A) & & & & & 25c & = & 1 \\
  (B) & & &10b & +& 5c & = & 1 \\
  (C) & a & +&b & +&c & = & 1 \\
\end{matrix*}
\] \[ \cdots \] \[
\begin{matrix*}[r]
  (A) & & & & & c & = & \frac{1}{25} \\
  (B) & & & b & & & = & \frac{2}{25} \\
  (C) & a & & & & & = & \frac{22}{25} \\
\end{matrix*}
\]

Es ergibt sich also als gemischte Strategie für Spieler II:
\[
\sigma^*_{II} = \sum_{s_{II}} \in S_{II} p(s_{II}) \text{ mit } p : S_{II} \mapsto [0, 1], p(L) = 
\frac{22}{25}, p(C) = \frac{2}{25}, p(R) = \frac{1}{25}
\]

Der erwartete Gewinn ist hierbei $-3 \cdot \frac{22}{25} + 3 \cdot \frac{2}{25} = -
\frac{66}{25} + \frac{6}{25} = -\frac{60}{25}$. \\

\begin{Large}
Aufgabe 4.2\\[0.0cm]
\end{Large}

a)

\[
a_1, a_2 \in A, b_1, b_2 \in B.
\]

Da $A$ und $B$ konvex:

\[
(\lambda a_1 + (1 - \lambda) \cdot a_2) \in A, (\lambda b_1 + (1 - \lambda) \cdot b_2) \in B
\]
\\
\[
(\lambda a_1 + (1 - \lambda) \cdot a_2, \lambda b_1 + (1 - \lambda) \cdot b_2)
\]
\[
= (\lambda a_1, \lambda b_1) + ((1 - \lambda) \cdot a_2, (1 - \lambda) \cdot b_2)
\]
\[
= \lambda(a_1, b_1) + (1 - \lambda) \cdot (a_2, b_2)
\]
\[
\Rightarrow \lambda(a_1, b_1) + (1 - \lambda) \cdot (a_2, b_2) \in A \times B \Rightarrow A \times
B \text{ konvex.}
\]



\end{document}

