\documentclass{scrartcl}
 
\usepackage[utf8]{inputenc}
\usepackage[T1]{fontenc}
\usepackage{lmodern}
\usepackage[pdftex]{graphicx}
\usepackage[ngerman]{babel}
\usepackage{amsmath}
\usepackage{amssymb}
\usepackage{tabularx}
\usepackage{multirow}
\usepackage{amsfonts}
\usepackage{tabto}
\usepackage{mathtools}
\TabPositions{0.1in, 0.4in, 0.6in, 0.8in, 1.0in, 1.2in, 3.4in}

\begin{document}
\begin{Large}
Aufgabe 11.2\\[0.0cm]
\end{Large}

a) \\

Wir versuchen, den erwarteten Nutzen für jeden Spieler zu maximieren. Wir nehmen dabei an,
dass $b_i \leq 2$, da es nicht sinnvoll ist, ein Gebot abzugeben, welches größer ist als der
mögliche Wert des Gegenstandes für einen Spieler. \\

Wir zeigen, dass folgende Strategie für jeden Spieler ein symmetrisches Gleichgewicht ist: \\

\[
b_i(v_i) = \frac{v_i}{2}, \quad i = 1, 2, 3
\]

Wir betrachten ohne Einschränkung den Nutzen für Spieler I und bezeichnen mit $b^*$ das
Gebot, welches abgesehen vom Gebot von Spieler I, das höchste Gebot sei (unter der
Annahme, dass diese auch diese Strategie spielen - ihre Gebote seien daher $b^*_2$ und
$b^*_3$): \\

\[
b^* = \frac{v^*}{2}= max \{b^*_2, b^*_3\} = max \{\frac{v_2}{2}, \frac{v_3}{2}\}
\]

Damit ist der Nutzen für Spieler 1: \\

\[
u_1(b_1, b^*, v_1) = u_1(b_1, \frac{v^*}{2}, v_1)
\]
\[
= P(b_1 > \frac{v^*}{2}) \cdot (v_1 - b_1)
\]
\[
= P(2b_1 > v^*) \cdot (v_1 - b_1)
\]
\[
= min\{2b_1,1\} \cdot (v_1 - b_1)
\]

Diese Funktion ist quadratisch auf dem Intervall $b_1 \in [0, \frac{1}{2}]$ (erreicht
das Maximum bei $b_1 = \frac{v_1}{2}$), und linear mit negativer Steigung, wenn
$b_1 > \frac{1}{2}$. \\

Fazit: Wenn die beiden anderen Spieler die Strategie $b_i = \frac{v_i}{2}$ spielen,
so ist für Spieler I $b_1 = \frac{v_1}{2}$ die beste Antwort. Damit ist $b^* =
b_i(v_i) = \frac{v_i}{2}$ ein symmetrisches Gleichgewicht.


\end{document}