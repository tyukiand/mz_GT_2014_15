\documentclass{scrartcl}
 
\usepackage[utf8]{inputenc}
\usepackage[T1]{fontenc}
\usepackage{lmodern}
\usepackage[pdftex]{graphicx}
\usepackage[ngerman]{babel}
\usepackage{amsmath}
\usepackage{amssymb}
\usepackage{tabularx}
\usepackage{multirow}
\usepackage{amsfonts}
\usepackage{tabto}
\usepackage{mathtools}
\TabPositions{0.1in, 0.4in, 0.6in, 0.8in, 1.0in, 1.2in, 3.4in}

\begin{document}
\begin{LARGE}
Spieltheorie - WiSe 2014/15
\end{LARGE}

\begin{Large}
Übungsblatt 10 - Felix Dosch\\[1.0cm]
\end{Large}

\begin{Large}
Aufgabe 10.1\\[0.0cm]
\end{Large}

a) \\

Die Strategie $C$ wird von $D$ strikt dominiert. Das heißt, es ist für jeden der Spieler (egal
welche gemischte Strategie gerade von ihm gespielt wird) rational, mehr $D$ zu spielen als er es gerade
tut (falls er nicht eh schon die reine Strategie $D$ spielt. \\

Es kann also außer \textsl{always defect} kein anderes Nash-Gleichgewicht geben, da einer der
Spieler seiner Nutzen immer erhöhen kann. Soweit für das einfache Spiel. \\

Im endlich oft iterierten Spiel nehmen wir nun an, es gebe ein Nash-Gleichgewicht, welches nicht
\textsl{always defect} sei. Da die Spieler wissen, welches die letzte Iteration ist, können sie
in dieser Iteration $D$ spielen, da sie nicht vom anderen dafür "`bestraft"' werden können. Daher
kann diese Strategie kein Nash-Gleichgewicht sein, da es sich für jeden der Spieler lohnnt,
abzuweichen. \\

b) \\

Da $AD^t_I = AD^t_{II} = D$ für alle $t$, folgt, dass: \\

\[
\lim_{T \to \infty} \frac{1}{T} \sum_{t=1}^{T} U(AD^t_I, AD^t_{II}) = \lim_{T \to \infty}
\frac{1}{T} \sum_{t=1}^{T} U(D, D) = \lim_{T \to \infty} \frac{1}{T} \sum_{t=1}^{T} 1
\]
\[
= \lim_{T \to \infty} \frac{1}{T} \cdot T = \lim_{T \to \infty} 1 = 1
\]

Weicht ein Spieler ab, so erhält er einen Nutzen von 0, der Gegner einen Nutzen von 4 in dieser
Iteration. Er kann nun weiter $C$ spielen und 0 erhalten, oder zurückkehren zu $D$ und wieder 1
bekommen, aber nicht mehr als 1 pro Durchgang. \\

c) \\

Im Gegensatz zu a) wäre im unendlich oft iterierten Spiel \textsl{always cooperate} ein Nash-
Gleichgewicht (Darauf läuft es hinaus, wenn beide Spieler \textsl{tit for tat} spielen und keiner
von $C$ abweicht)$^1$. Das Abweichen und der einmalige Gewinn von 4 zahlen sich über lange Sicht
nicht aus:

I weicht von $C$ ab und spielt $D$, während II noch $C$ spielt. In der nächsten Runde spielt
nun I $C$ (was II zuvor gespielt hat) und II spielt $D$ (was I zuvor gespielt hat). Dies
wiederholt sich offensichtlich von diesem Zeitpunkt an. Damit hat mal Spieler I einen Nutzen von
4, II einen Nutzen von 0, mal umgekehrt. Im Schnitt kommt jeder Spieler also auf einen Nutzen
von 2. Im unendlich oft wiederholten Spiel wäre der durchschnittliche Nutzen für jeden Spieler
also 2. \\

Tit for Tat ohne Abweichung: \\

\[
\lim_{T \to \infty} \frac{1}{T} \sum_{t=1}^{T} U(TFT^t_I, TFT^t_{II}) \overset{1}= \lim_{T \to \infty} 
\frac{1}{T} \sum_{t=1}^{T} U(C, C) = \lim_{T \to \infty} \frac{1}{T} \sum_{t=1}^{T} 3 = 3
\]

Tit for Tat mit Abweichung von Spieler I in Iteration $k$, Nutzen für I: \\

\[
\lim_{T \to \infty} \frac{1}{T} \sum_{t=1}^{k-1} U_I(C, C) + \sum_{t = k+1}^{T}
U_I(TFT'^t_I, TFT'^t_{II}) = \lim_{T \to \infty} \frac{1}{T} ( 3 \cdot k + \sum_{t = k+1}^{T}
2)
\]
\[
= \lim_{T \to \infty} \frac{1}{T} \cdot 3 \cdot k + \lim_{T \to \infty} \frac{1}{T}
\sum_{t = k+1}^{T} 2 = 0 + \lim_{T \to \infty} \frac{1}{T} \cdot (T - k) \cdot 2 = 0 + 2 = 2
\]

Für Spieler II verhält sich der Grenzwert genau so. Ebenso könnte es Spieler II sein, der abweicht.
Da der Erwartungswert (wenn keiner abweicht) höher ist, als wenn einer der Spieler abweicht, ist
\textsl{tit for tat} ein Nash-Gleichgewicht im unendlich oft iterierten Spiel.

\end{document}