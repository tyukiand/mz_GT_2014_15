\documentclass{scrartcl}
 
\usepackage[utf8]{inputenc}
\usepackage[T1]{fontenc}
\usepackage{lmodern}
\usepackage[pdftex]{graphicx}
\usepackage[ngerman]{babel}
\usepackage{amsmath}
\usepackage{amssymb}
\usepackage{tabularx}
\usepackage{multirow}
\usepackage{amsfonts}
\usepackage{tabto}
\usepackage{mathtools}
\TabPositions{0.1in, 0.4in, 0.6in, 0.8in, 1.0in, 1.2in, 3.4in}

\begin{document}
\begin{LARGE}
Spieltheorie - WiSe 2014/15
\end{LARGE}

\begin{Large}
Übungsblatt 5 - Felix Dosch\\[1.0cm]
\end{Large}

\begin{Large}
Aufgabe 5.1\\[0.0cm]
\end{Large}

a) Zu zeigen:
\[
\Sigma_i = \Delta(S_i) = \{\sigma : S_i \longrightarrow [0,1] | \sum_{s_i \in S_i} \sigma(s_i) = 1\}
\]
ist kompakt. Dazu benutzen wir den Satz von Heine-Borel und zeigen
\begin{itemize}
\item{i) $\Sigma_i$ ist abgeschlossen}
\item{ii) $\Sigma_i$ ist beschränkt}
\end{itemize}

Zu i) \\

Zu zeigen: 
\[
\forall x \in \mathbb{R}^n \setminus \Sigma_i \quad  \exists \epsilon > 0 \quad \forall y 
\in \mathbb{R}^n \text{ mit } \|x-y\| < \epsilon : y \in \mathbb{R}^n \setminus \Sigma_i
\], wobei $n$ die Anzahl der Spieler sein möge.
In Worten: Für jeden Punkt ausserhalb von $\Sigma_i$ gibt es eine Umgebung mit "Radius" $\epsilon$,
sodass jeder Punkt in dieser Umgebung ebenfalls ausserhalb von $\Sigma_i$ liegt. \\

Wir betrachten die Elemente von $\Sigma_i$ als Vektoren des $\mathbb{R}^n$:
\[
\Sigma_i =  \{v = (v_1, v_2, ..., v_n)^T \in \mathbb{R}^n | \sum_{i = 1}^n v_i = 1, v_i \geq 0 \quad 
\forall i\}
\]
 $\Sigma_i$ ist Teilmenge
einer Hyperebene im $\mathbb{R}^n$, welche durch die Punkte $(1, 0, ..., 0)$, $(0, 1, ..., 0)$ ...
$(0, 0, ..., 1)$ verläuft. Der Normalenvektor zu dieser Hyperebene ist der Vektor $\overrightarrow{n}
(\frac{1}{n}, \frac{1}{n}, ..., \frac{1}{n})^T$.

Die Entfernung eines beliebigen Punktes in $\mathbb{R}^n \setminus \Sigma_i$ zu $\Sigma_i$ ist also
in dieser Richtung (gleichmäßig in jeder Dimension) am kleinsten. \\

Sei nun $x \in \mathbb{R}^n \setminus \Sigma_i$ beliebig mit $\sum_{i \in \{1, ..., n\}} x_i = 1 + 
\delta$. Per Definition von $\Sigma_i$ ist $\delta \neq 0$.

Wir verteilen $\delta$ gleichmäßig
auf alle Dimensionen "`auf dem Weg"' zu einem Element in $\Sigma_i$. Wie oben argumentiert, erreichen wir
ein solches mit minimalem Abstand in dieser Richtung.

Wir betrachten den Punkt $y$, den wir erreichen, mit

\[\|x - y\| = \sqrt{(\frac{\delta}{n})^2 + (\frac{\delta}{n})^2 + ... + (\frac{\delta}{n})^2} = 
\sqrt{n \cdot (\frac{\delta}{n})^2} = \sqrt{n} \cdot \frac{\delta}{n} = \frac{\delta}{\sqrt{n}}
\]

$y$ ist entweder in $\Sigma_i$ oder nicht, es gibt aber keinen Punkt, der näher an $x$ liegt und in
$\Sigma_i$ ist (wg. Normalenvektor).

Daher: 

Für jedes $x = (1 + \delta) \in \mathbb{R}^n \setminus \Sigma_i$ wähle $\epsilon = \frac{\delta}
{\sqrt{n}}$, womit $\forall y \in \mathbb{R}^n \text{ mit } \|x-y\| < \epsilon : y \in \mathbb{R}^n 
\setminus \Sigma_i$. \\

Zu ii) \\

Wir betrachten $\Sigma_i$ wieder als Menge von Vektoren des $\mathbb{R}^n$. Wir können die Vektoren nach
ihrer Länge ordnen und eine obere und untere Schranke angeben, womit die Menge beschränkt ist.

Als obere Schranke kann man 1 wählen (Einheitsvektoren in jede Richtung).

Als untere Schranke kann man den Abstand der Hyperebene zum Ursprung wählen, also $\frac{1}{\sqrt{n}}$. \\

\begin{Large}
Aufgabe 5.2\\[0.0cm]
\end{Large}

a) $X$ ist nicht leer, da $\Sigma_i$ nicht leer für alle $i$. $X \subseteq \mathbb{R}^m$, da $\Sigma_i
\subseteq \mathbb{R}^l \forall i$ und $X = \bigtimes_{i} \Sigma_i \Rightarrow X \subseteq 
\mathbb{R}^{n \times l} = \mathbb{R}^m$. $X$ ist als Kreuzprodukt konvexer und kompakter Mengen ebenfalls
kompakt und konvex (in Aufgabe 2.4 bewiesen). \\

b) Zu zeigen: Die Menge der "`besten Antworten"' ist nicht leer und konvex. \\

\begin{itemize}
\item{nicht leer: $U_i$ liefert immer einen Wert zurück, also ist für $arg max$ immer mindestens ein Wert
verfügbar, der zurückgegeben werden kann. Also ist die zurückgegebene Menge nicht leer.}
\item{Konvexität: Entweder gibt es genau eine Strategie, die die beste Antwort ist (Ein-elementige Menge
trivialerweise konvex) oder es gibt mehrere gemischte Strategien als beste Antworten - Dann sind auch alle
möglichen Konvex.Kombinationen zweier solcher Strategien Elemente der besten Antworten. (Es ist egal wie Spieler
$i$ seine Strategie wählt, es hat keinen Einfluss auf das Ergebnis)}
\end{itemize}

c) Zu zeigen: Der Graph von $f$ ist abgeschlossen. \\

Wie das zu zeigen ist, weiß ich nicht. Das einzige was ich jetzt dazu gefunden habe ist folgendes, was sich
aber nicht auf den Graph von $f$, sondern auf $f$ bezieht: \\

Die Abgeschlossenheit von $BR_i(\sigma)$ folgt aus der Stetigkeit der $U_i$, denn für jede konvergente
Folge $^{k}t \in U_i(\sigma'_i, \sigma_{-i})(s)$ folgt aus der Ungleichung $U_i(^{k}t^{i}, \sigma_{-i})
\geq U_i(\sigma'_i, \sigma_{-i})$ die Ungleichung im Grenzpunkt.\footnote{
https://www.uni-due.de/\~{}hn215go/gollmer/teaching/Spieltheorie\_3.pdf - S.6, Beweis zu
Lemma 3.96}

d) Da alle Voraussetzungen des Satzes erfüllt sind (wie in a) bis c) gezeigt), hat $BR_i$ einen Fixpunkt.
Das heißt $\exists \sigma'_i \in \Sigma_i : \sigma'_i \in BR_i(\sigma'_i, \sigma_{-i})$. In anderen
Worten: Es gibt eine Kombination gemischter Strategien der anderen Spieler, sodass bei Wahl von $\sigma'_i$
als Strategie von Spieler $i$, seine beste Antwort $\sigma'_i$ ist, bzw. keine andere Antwort besser ist.

Da dies nicht nur für einen Spieler, sondern für alle Spieler gilt (betrachte $BR$ statt nur $BR_i$), hat
kein Spieler eine bessere Antwort als seine aktuell gewählte Strategie. Daher besitzt das Spiel ein
Nash-Gleichgewicht in gemischten Strategien. \\

\begin{Large}
Aufgabe 5.3\\[0.0cm]
\end{Large}

Das einzige symmetrische Nash-Gleichgewicht in diesem Spiel ist $((0 \cdot Dove, 1 \cdot Hawk),(0 \cdot
Dove, 1 \cdot Hawk))$, da bei(Dove, Dove) z.B. in der Zeile eine Abweichung in Richtung Hawk einen
Nutzenzuwachs bringt. \\

Im Gleichgewicht (Hawk, Hawk) ist bei einer Abweichung um $\epsilon$ (Störung) Hawk immer noch die beste
Antwort, sowohl für den Zeilenspieler (bei Abweichung des Spaltenspielers), als auch umgekehrt für den
Spaltenspieler (aufgrund der Symmetrie). Insbesondere gilt dies, da für alle reinen Strategien gilt, dass
der Nutzen bei Wahl von Hawk IMMER größer ist als der Nutzen von Dove - unabhängig von der Strategie des
Gegners. Es ist also IMMER die beste Strategie, Hawk zu spielen, somit ist es auch immer die beste Antwort
und damit natürlich auch bei geringer Störung des Spiels. Daher ist (Hawk, Hawk) auch evolutionär
stabile Strategie. \\ \\

\Large {Anmerkung:} \\

\normalsize
Ich habe den Eindruck, dass die Aufgaben nochmal etwas schwerer geworden sind - entgegen deiner Aussage,
dass die Aufgaben etwas leichter werden sollten. Normalerweise habe ich keine übermäßig großen Probleme mit 
mathematischen Aufgaben, jedoch habe ich den Eindruck, dass mein mathematisches Wissen hier und da bei den
Aufgaben an seine Grenzen stößt...Ich habe den Mittwoch-Nachmittag und fast den gesamten Donnerstag und nun
noch Montag-Abend mit der Bearbeitung der Aufgaben verbracht. Ich finde das etwas viel.
\end{document}

