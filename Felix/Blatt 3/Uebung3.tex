\documentclass{scrartcl}
 
\usepackage[utf8]{inputenc}
\usepackage[T1]{fontenc}
\usepackage{lmodern}
\usepackage[pdftex]{graphicx}
\usepackage[ngerman]{babel}
\usepackage{amsmath}
\usepackage{amssymb}
\usepackage{tabularx}
\usepackage{multirow}
\usepackage{amsfonts}
\usepackage{tabto}
\TabPositions{0.1in, 0.4in, 0.6in, 0.8in, 1.0in, 1.2in, 3.4in}

\begin{document}
\begin{LARGE}
Spieltheorie - WiSe 2014/15
\end{LARGE}

\begin{Large}
Übungsblatt 3 - Felix Dosch, Julian Felix Rost\\[1.0cm]
\end{Large}

\begin{Large}
Aufgabe 3.2\\[0.0cm]
\end{Large}

Zu zeigen: Eine strikt dominierte Strategie ist niemals Teil eines Nash-Gleichgewichts. \\

Beweis durch Widerspruch: O.B.d.A. betrachten wir Strategien für Spieler I. Sei $s_I$
eine strikt dominierte Strategie. Das heißt, $\exists t_I \in S_I$ mit

\begin{align}
u_I(s_I,s_{-I}) < u_I(t_I,s_{-I}) \quad \text{für alle } s_{-I} \in S_{-I}
\end{align}

Annahme: $s_I$ ist Teil eines Nash-Gleichgewichts, das heißt

\[
\exists s^*_{-I} \in S_{-I} : u_I(s_I,s^*_{-I}) \geq u_I(s'_I,s^*_{-I}) \quad 
\text{für alle } s'_I \in S_I
\]

Jedoch (1):

\[
\exists t_I \in S_I : u_I(s_I,s^*_{-I}) \ngeq u_I(t_I,s^*_{-I})
\]

Dies steht in Widerspruch zur Annahme, dass $s_I$ Teil eines Nash-Gleichgewichtes ist. \\

\begin{Large}
Aufgabe 3.3\\[0.0cm]
\end{Large}

Wir betrachten das folgende Spiel in Strategieform: \\

\begin{tabularx}{0.7\textwidth} {|X|X|X|X|}
\hline
$\downarrow$ I / II $\rightarrow$	& $A$		& $B$		& $C$	\\
\hline
$a$									& (10,4)	& (7,7)		& (0,24)\\
\hline
$b$									& (9,8)		& (4,11)	& (10,10)\\
\hline
$c$									& (6,2)		& (3,9)		& \textbf{(10,10)}\\
\hline
\end{tabularx} \\

\textbf{Nash-Gleichgewicht: }Der Strategievektor $s^* = ((c),(C))$ ist das einzige Nash-Gleichgewicht 
in diesem Spiel. Für Spieler II lohnt es sich nicht, seine Strategie zu ändern (sein Nutzen würde auf
9, bzw. 2 reduziert). Für Spieler I lohnt es sich ebenfalls nicht (seine Nutzen würde gleich bleiben
oder auf 0 sinken).

Der Strategievektor $((b), (C))$ ist kein Nash-Gleichgewicht, obwohl er für beide Spieler den gleichen
Nutzen hat wie $s^*$. Jedoch könnte hier II mit einem Wechsel zu Strategie $(B)$ seinen Nutzen
verbessern. \\

\textbf{Iterative Eliminierung schwach dominierter Strategien: }

\begin{itemize}
\item{Spieler I kann Strategie $(c)$ eliminieren, da diese von $(b)$ schwach dominiert wird (9 > 6,
4 > 3, 10 = 10). Damit verschwindet das ursprüngliche Nash-Gleichgewicht.}
\item{Spieler II kann Strategie $(A)$ eliminieren, da diese von $(B)$ sogar strikt (und damit natürlich
auch schwach) dominiert wird (7 > 4, 11 > 8).}
\end{itemize}

Es verbleibt das folgende Spiel: \\

\begin{tabularx}{0.5\textwidth} {|X|X|X|}
\hline
$\downarrow$ I / II $\rightarrow$	& $B$		& $C$	\\
\hline
$a$									& (7,7)		& (0,24)\\
\hline
$b$									& (4,11)	& (10,10)\\
\hline
\end{tabularx} \\

Dieses Spiel besitzt kein Nash-Gleichgewicht. Die Strategien können im Uhrzeigersinn gewechselt werden,
wobei sich für den wechselnden Spieler ein Vorteil ergibt. \\

\begin{Large}
Aufgabe 3.4\\[0.0cm]
\end{Large}

a) Da sich der Gewinn für Spieler $i$ wie in der Vorlesung berechnet (falls $i$ den Gegenstand
bekommt aus der Differenz des Gegenstandswertes für den Spieler $v_i$ und seinem Gebot $b_i$,
ansonsten 0) ergibt sich für den Spieler $i$ beim Spielen der Strategie $b_i = v_i$ immer ein Gewinn 
von 0.

Entweder ersteigert er den Gegenstand nicht, oder er muss genau $b_i$ dafür bezahlen, womit er auch
einen Gewinn von 0 hat. \\

Allerdings dominiert $b_i = v_i$ nicht alle anderen Strategien schwach, denn: \\

Bietet $i$ nicht $v_i$ sondern einen kleineren Wert $b_i = v_i^*$ mit $0 < v_i^* < v_i$, so ergibt
sich für den Fall, dass alle anderen Spieler ein Gebot abgeben, welches kleiner als $v_i^*$ ist ein
Gewinn von $v_i - v_i^* > 0$. \\

b) siehe Aufgabe a). \\

\begin{Large}
Aufgabe 3.5\\[0.0cm]
\end{Large}

a) Zu zeigen: \\
\[
\min_{a \in A} \max_{b \in B} u(a,b) \geq \max_{b \in B} \min_{a \in A} u(a,b)
\]
Dazu zeigen wir zunächst folgendes:
\[
\max_{b \in B} u(x,b) \geq u(x,y) \geq \min_{a \in A} u(a,y) \quad \text{mit }  x \in A, y \in B
\]

Die erste Ungleichung gilt, da der Nutzen für festes $x$ mit $b$ maximiert wird, d.h. $u(x,y)$ kann
für freies $y$ höchstens so groß werden, wie für $y = b$. Bei der zweiten Ungleichung ist das
symmetrisch für festes $y$ und freies $x$ der Fall. \\

Daher:

\[
\max_{b \in B} u(x,b) \geq \min_{a \in A} u(a,y)
\]
\[
\Leftrightarrow \min_{a \in A} \max_{b \in B} u(a,b) \geq \min_{a \in A} u(a,y)
\]
\[
\Leftrightarrow \min_{a \in A} \max_{b \in B} u(a,b) \geq \max_{b \in B} \min_{a \in A} u(a,b).
\]

Die Ausdrücke sind wohldefiniert, da z.B. $\min_{a \in A} u(a,\_)$ die Menge aller $u \in \mathbb{R}$
liefert, die mögliche Ergebnisse bei Wahl von $a$ beinhaltet, wovon durch $\max_{b \in B} \ldots$
das Maximum, also ein Wert $\in \mathbb{R}$ definiert wird. Die Relation $\geq$ ist auf
$\mathbb{R}$ definiert. \\

b) Zu zeigen: In einem Nullsummenspiel für zwei Spieler gilt:

\[
\min_{s_I \in S_I} \max_{s_{II} \in S_{II}} u_{II}(s_I,s_{II}) = - \max_{s_I \in S_I} \min_{s_{II} 
\in S_{II}} u_I(s_I,s_{II})
\]

Dazu benutzen wir die Definition des Nullsummenspiels für 2 Spieler:

\[
\min_{s_I \in S_I} \max_{s_{II} \in S_{II}} u_{II}(s_I,s_{II}) = \min_{s_I \in S_I} \max_{s_{II} 
\in S_{II}} - u_{I}(s_I,s_{II})
\]
\[
= \min_{s_I \in S_I} ( -\min_{s_{II} \in S_{II}} u_{I}(s_I,s_{II}))
\]
\[
= - \max_{s_I \in S_I} \min_{s_{II} \in S_{II}} u_{I}(s_I,s_{II}).
\] \\

c) Ein Nullsummen-Spiel, für welches die Gleichung gilt, ist einfach anzugeben, da auf beiden
Seiten der Gleichung der Gewinn von Spieler II maximiert wird, dabei ist die Reihenfolge
egal. Wir geben $u_I(s_I,s_{II})$ an. Dabei ist nur darauf zu achten, dass der erreichte Gewinn
für beide gleich ist, d.h. wenn alle anderen Werte der Tabelle größer sind als $x$, dann ist
$x$ der Wert, der jeweils für die Ausdrücke links und rechts der Gleichung heraus kommt. Um die
Gleichung zu erfüllen muss dieser 0 sein: \\

\begin{tabularx}{0.5\textwidth} {|X|X|X|}
\hline
$\downarrow$ I / II $\rightarrow$	& $A$		& $B$	\\
\hline
$a$									& 0		& 	2\\
\hline
$b$									& 2		& 	4\\
\hline
\end{tabularx} \\

Nun ist es nicht schwer, das Spiel so zu ändern, dass die Gleichung nicht mehr stimmt:

\begin{tabularx}{0.5\textwidth} {|X|X|X|}
\hline
$\downarrow$ I / II $\rightarrow$	& $A$		& $B$	\\
\hline
$a$									& \textbf{1}		& 	2\\
\hline
$b$									& 2		& 	4\\
\hline
\end{tabularx} \\


\end{document}

