\documentclass{scrartcl}
 
\usepackage[utf8]{inputenc}
\usepackage[T1]{fontenc}
\usepackage{lmodern}
\usepackage[pdftex]{graphicx}
\usepackage[ngerman]{babel}
\usepackage{amsmath}
\usepackage{tabularx}
\usepackage{multirow}
\usepackage{amsfonts}
\usepackage{tabto}
\TabPositions{0.1in, 0.4in, 0.6in, 0.8in, 1.0in, 1.2in, 3.4in}

\begin{document}
\begin{LARGE}
Spieltheorie - WiSe 2014/15
\end{LARGE}

\begin{Large}
Übungsblatt 2 - Felix Dosch, Julian Felix Rost\\[1.0cm]
\end{Large}

\begin{Large}
Aufgabe 2.1\\[0.0cm]
\end{Large}

a) 

\begin{itemize}
\item{Spiel 1:}
\begin{itemize}
\item{I weiß nicht, wie das initiale Zufallsexperiment ausgegangen ist}
\item{II weiß weder, wie das initiale Zufalssexperiment ausgegangen ist, noch welche Wahl I getroffen hat}
\end{itemize}
\item{Spiel 2:}
\begin{itemize}
\item{II kennt den Ausgang des Zufallsexperiments - in einem Fall kann II zwischen $t_1$ und $b_1$ wählen,
im anderen Fall zwischen $t_2$ und $b_2$}
\item{I kennt den Ausgang des Zufallexperiments nicht, da II entweder mit $t_1$ oder $t_2$ gewählt haben kann}
\end{itemize}
\end{itemize}

b)
\begin{itemize}
\item{Spiel 1:}
\end{itemize}

\begin{tabularx}{0.5\textwidth} {|X|X|X|}
\hline
$\downarrow$ I / II $\rightarrow$	& $T$		& $B$	\\
\hline
$T$					& (5,4)			& (9,4)	\\
\hline
$B$					& (2,8)			& (8,3)	\\
\hline
\end{tabularx}

\begin{itemize}
\item{Spiel 2:}
\end{itemize}

\begin{tabularx}{1\textwidth} {|X|X|X|X|X|}
\hline
$\downarrow$ I / II $\rightarrow$	& $t_1t_2$		& $t_1b_2$		& $b_1t_2$		& $b_1b_2$	\\
\hline
$T_1$								& (40,22)		& (36,26)		& (16,4)		& (12,8)		\\
\hline
$B_1$								& (44,24)		& (48,20)		& (8,12)		& (12,8)		\\
\hline
\end{tabularx}
\clearpage

\begin{Large}
Aufgabe 2.2\\[0.0cm]
\end{Large}

Behauptung: Das Ergebnis einer iterierten Eliminierung strikt dominierter Strategien ist unabhängig von 
der Reihenfolge der einzelnen Eliminierungen. \\

Beweis:  Eine iterierte Eliminierung strikt dominierter Strategien besteht notwendigerweise aus dem
Verfahren der Eliminierung und einer Menge strikt dominierter Strategien. 

Für alle diese strikt dominierten Strategien gilt, dass es eine Strategie mit maximalem Nutzen im Verhältnis 
zu ihren Alternativen und allen möglichen Gegenstrategien unter ihnen gibt, da die Definition von strikt
dominierten Strategien vorgibt, dass es eine Bijektion vom Raum des Kreuzprodukts der Strategien und allen 
Gegenstrategien gibt zu einem totalgeordneten Raum des Nutzen (zumindest gehen wir davon aus, dass dieser 
Raum totalgeordnet ist). 

Somit sind Paare aus Strategien und allen möglichen Gegenstrategien ebenfalls totalgeordnet. 
Da alle möglichen Gegenstrategien in jedem dieser Paare verhältnismäßig äquivalent sind, weil es jeweils 
alle sind, können wir genauso gut sagen, dass alle Strategien totalgeordnet sind. 

Daher muss es irrelevant sein, in welcher Reihenfolge schwächere Strategien eliminiert werden; da immer nur
 schwächere, und nicht stärkere eliminiert werden, und immer eine maximal dominierende Strategie existiert 
(aufgrund der Existenz einer Totalordnung), die also immer als Ergebnis erreicht werden muss. 

\clearpage

\begin{Large}
Aufgabe 2.3\\[0.0cm]
\end{Large}

a) Zu zeigen: Das Chomp-Spiel endet auf einem ($\infty \times \infty$)-Feld stets nach endlich
vielen Zügen. \\

Zum Beweis wollen wir im Folgenden die Spalten und Reihen jeweils in zwei verschiedene Kategorien
einteilen: In begrenzte Spalten/Reihen $B = B_C$ $\dot{\bigcup}$ $B_R$ (endlich viele Felder frei in
dieser Spalte/Reihe) und unbegrenzte Spalten/Reihen $U = U_C$ $\dot{\bigcup}$ $U_R$. Für alle
$c \in B_C$ und $r \in B_R$ bezeichne $l(c)$, bzw. $l(r)$ die Anzahl belegbarer Felder in dieser
Spalte/Reihe. Da das Spiel endet, wenn einem Spieler nur noch (1,1) als mögliches Feld bleibt und
daher auch keine unbegrenzte Spalte oder Reihe mehr existiert ($|U| = 0$), wollen wir zeigen, dass
wir $|U| = 0$ und in der Folge das Ende des Spiels in einer endlichen Anzahl von Zügen erreichen.\\

Nach dem ersten Zug von $I$ auf beliebiges Feld $(a,b)$ ergeben sich $B_C$, $B_R$, $U_C$ und $U_R$ 
und es gilt: $\forall c \in B_C : l(c) = b-1 \bigwedge \forall r \in B_R : l(r) = a - 1$ (alle
begrenzten Spalten haben die gleiche Anzahl freier Felder, analog für Reihen (I)). Weiter gilt
$|U| = |U_C| + |U_R| = (a-1) + (b-1) = a + b - 2$.

Der nachfolgende Zug verringert $|U|$ zwangsläufig, da entweder eine Spalte oder eine Reihe (oder
beides) begrenzt werden (II). \\

Nach jedem Zug gibt es zwei Möglichkeiten:
\begin{itemize}
\item{i) Es entsteht eine Situation wie in I, wodurch $|U|$ anschließend kleiner wird (II)}
\item{ii) Es entsteht eine Situation, in der begrenzte Reihen und Spalten nicht alle jeweils
gleich viele Felder haben ($\exists c_1, c_2 \in B_C : l(c_1) \neq l(c_2)) \bigwedge \exists 
r_1, r_2 \in B_R : l(r_1) \neq l(r_2)$)}
\end{itemize}

Im zweiten Fall gilt Folgendes: Ein Zug auf ($\min_{r \in B_R}{l(r)}+1,\min_{c \in B_C}{l(c)}+1)$
\begin{itemize}
\item{verringert $|U|$ nicht}
\item{führt wieder zu einer Situation wie in I}
\item{eliminiert eine endliche Anzahl belegbarer Felder, sprich es existiert  eine endliche Anzahl 
an möglichen Zügen, bevor dieses Feld erreicht wird und $|U|$ dann verringert werden muss (III)}
\end{itemize}

Aus der Tatsache, dass $|U|$ mit jedem Anfangszug einen endlichen Wert annimmt und spätestens nach
endlich vielen Zügen jeweils verringert wird folgt, dass $|U|$ nach einer endlichen Anzahl von
Zügen den Wert 0 annimmt. In diesem Fall bleibt nur eine endliche Anzahl von Feldern übrig, was
eine endliche Anzahl von verbleibenden Zügen impliziert, welche nach jedem Zug ebenfalls kleiner
wird, bis kein Zug mehr möglich und das Spiel beendet ist. \\

b) Auf dem ($\infty \times 2$)-Feld hat Spieler 2 eine Gewinnstrategie. Diese lässt sich wie folgt
grob beschreiben:

Spieler I kann auf ein Feld in der oberen oder in der unteren Reihe setzen.
\begin{itemize}
\item{Setzt I in die untere Reihe, so entsteht nach diesem Zug ein ($m \times 2$)-Feld. Für dieses
endliche, rechteckige Feld wurde in der Vorlesung gezeigt, dass für den Spieler, der hier den
ersten Zug machen kann, eine Gewinnstrategie existiert. Da Spieler II nun dran ist, besitzt er
eine Gewinnstrategie, falls I in die untere Reihe setzt und ein ($m \times 2$)-Feld ensteht}
\item{Setzt I in die obere Reihe auf $(m, 2)$, so ist die Strategie für II ein Setzen auf 
$(m+1, 1)$, d.h. in die untere Reihe, so, dass das Feld unter $(m, 2)$ frei bleibt. Die Konsequenz
ist, dass I mit einem Setzen in die untere Reihe II wieder eine Gewinnstrategie liefert (Fall 1) und
beim Setzen in die obere Reihe II wieder entsprechend reagiert (Fall 2), wobei das Spielfeld immer
kleiner wird.}
\end{itemize}

Offensichtlich kann I nur so lange in die obere Reihe setzten, bis auf Feld $(1,2)$ gesetzt wird. Dann
setzt II auf $(2,1)$ und es bleibt kein Zug für I. \\

c) Auf dem ($\infty \times n$)-Feld für $n \in \mathbb{N} \setminus \{2\}$ hat Spieler I eine
Gewinnstrategie, wobei wir 2 Fälle unterscheiden:
\begin{itemize}
\item{$n = 1$ : Trivial, hier setzt I auf $(2,1)$ und es bleibt nur $(1,1)$ für II}
\item{$n > 2$ : Spieler I setzt auf $(1,3)$ und führt das Spiel damit auf die Ausgangslage des
Spiels auf dem ($\infty \times 2$)-Feld zurück, wo der reagierende Spieler (wie in b) gezeigt) eine 
Gewinnstrategie hat. Da II jetzt an der Reihe ist, hat I somit eine Gewinnstrategie}
\end{itemize}

d) Auf dem ($\infty \times \infty$)-Feld hat Spieler I eine Gewinnstrategie. Insgesamt sind es genau
3 Gewinnstrategien. Wir geben zunächst die Gewinnstrategien an, bzw. zeigen dass diese existieren,
und zeigen dann, warum diese 3 die einzigen sind. Diese 3 Züge sind jeweils der Anfang einer
Gewinnstrategie für I:
\begin{itemize}
\item{$(1,3)$ : Dies führt das Spiel wie in c) zurück auf das Spiel auf dem ($\infty \times 2$)-Feld,
wofür wir in b) die Existenz einer Gewinnstrategie für den reagierenden Spieler gezeigt haben}
\item{$(3,1)$ : Symmetrisch zu $(1,3)$}
\item{$(1,1)$ : Die Gewinnstrategie auf dem quadratischen Spielfeld - nach diesem Zug kann I immer 
symmetrisch zu II ziehen, bis II auf $(2,1)$ oder $(1,2)$ ziehen muss}
\end{itemize}

Dass der 3. Fall genau eine Strategie beschreibt ist offensichtlich, zu zeigen ist noch, dass $(1,3)$
und $(3,1)$ die einzig möglichen ersten Züge einer Gewinnstrategie sind und dass die in
b) beschriebene Strategie für das ($\infty \times 2$)-Feld die einzige Gewinnstrategie ist. \\

Macht I keinen der oben genannten Züge, so zieht I entweder auf $(2,1)$ oder $(1,2)$ womit II leicht
direkt gewinnt, oder ermöglicht II mindestens einen dieser Züge, womit II eine Gewinnstrategie hat.
Daher bleiben nur diese Züge als Anfang einer Gewinnstrategie.

Wir sind also nun auf dem ($\infty \times 2$)-Feld\footnote{Wir zeigen hier nur den Fall, in dem I auf
$(1,3)$ gesetzt hat. Der $(3,1)$-Fall ist komplett symmetrisch dazu} mit II am Zug. Dass II dann mit 
einem Zug in die untere Reihe keine Gewinnstrategie mehr besitzt haben wir gezeigt. Auf einen Zug in 
die obere Reihe auf $(m,2)$ könnte I auf verschiedene Arten reagieren:
\begin{itemize}
\item{Zug in die obere Reihe : Jetzt kann II den Spieß umdrehen und die ursprüngliche Gewinnstrategie
für I anwenden, d.h. I verliert hier seine Gewinnstrategie}
\item{Zug in die untere Reihe auf $(a,1)$ mit $a \leq m \Rightarrow$
rechteckiges Feld, Spieler am Zug hat Gewinnstrategie, also verliert auch hier I seine Gewinnstrategie}
\item{Zug in die untere Reihe auf $(a,1)$ mit $a > m+1 \Rightarrow$ hier
kann II mit einem Zug auf $(b,1)$ mit $b=m+1$ zur Gewinnstrategie kommen, I verliert auch hier seine
Gewinnstrategie}
\item{Zug in die untere Reihe auf $(a,1)$ mit $a = m+1$. Dies ist der einzige verbleibende Zug und der
einzige, mit dem I eine Gewinnstrategie behält}
\end{itemize}
Daher hat Spieler I zu Beginn des Spiels genau 3 Gewinnstrategien.

\clearpage

\begin{Large}
Aufgabe 2.4\\[0.0cm]
\end{Large}

a) Behauptung: Ein zu Beginn schon auf dem Feld liegender Stein kann dem Spieler "`nie schaden"', dem er
gehört. \\

Zunächst wollen wir definieren, was wir unter "`schaden"', bzw. "`nie schaden"' verstehen. Ein Stein, der zu
Beginn auf einem Feld liegt, schadet - im Vergleich zum leeren Brett - dem Spieler, dem er gehört, wenn
dieser Spieler auf dem leeren Brett eine Gewinnstrategie hat und nun, mit dem Stein, keine Gewinnstrategie
mehr hat. Hatte der Spiele auf dem leeren Brett bereits keine Gewinnstrategie, so kann der Stein ihm
folglich nie schaden (nun hätte er schlimmstenfalls immer noch keine Gewinnstrategie).

Beweis:  Angenommen, es gäbe eine Gewinnstrategie für diesen Spieler ($A$) auf dem leeren Brett.\\

Fallunterscheidung: 
\begin{itemize}
\item{1. Fall: Der Stein liegt auf einem Feld, das in der Strategie nicht vor kommt. Dann kann der Spieler
$A$ einfach seine Strategie wie vorgesehen spielen, und der Stein hat für diese Strategie keinen Nutzen; 
insbesondere erzeugt er also auch keine negativen Auswirkungen. Die Tatsache, dass der Gegner $B$ nun ein
Feld weniger zur Auswahl hat und auf andere Felder ausweichen muss, ist nicht von Belang - könnte er mit
einem Zug auf eines dieser Felder einen Sieg von $A$ verhindern, dann hätte er dies auch bereits in der
Ausgangssituation mit dem leeren Brett tun können - laut Annahme hat $A$ hier jedoch eine
Gewinnstrategie. Somit hat $A$ immer noch eine Gewinnstrategie falls einer seiner Steine zu Beginn schon
auf dem Brett liegt.}
\item{2. Fall: Der Stein liegt auf einem Feld, das in der Strategie erwähnt wird. Dann kann der Spieler 
statt auf dieses Feld zu legen, auf ein weiteres, zufälliges Feld legen, und das Spielfeld im Weiteren
betrachten, als ob er seinen Zug hätte tätigen können. Für das zufällig bestimmte Feld gilt entweder Fall
1 oder Fall 2. Da das Spielfeld endlich ist, wird ein so kaskadiert belegtes Feld irgendwann ein Feld vom
Typ Fall 1 ($A$ hat noch Gewinnstrategie) sein, oder das Spiel nach dem Zug zu Ende sein, wobei $A$ gewinnt
da es bis zu diesem Zug eine Gewinnstrategie hat. Somit hat $A$ auch hier immer noch eine Gewinnstrategie
falls einer seiner Steine zu Beginn schon
auf dem Brett liegt.}
\end{itemize}

b)

i) Es reicht, ein voll besetztes Brett zu betrachten, da bei noch nicht voll besetzten Brettern entweder
bereits ein Spieler gewonnen hat, oder ein Spieler noch am Zug ist, d.h. das Spiel ist noch nicht beendet.\\

ii) Wir zeigen das Lemma durch Betrachtung der Knotengrade. Laut Voraussetzung haben alle Knoten höchstens
den Grad 2.

\begin{itemize}
\item{Grad 0 : \textbf{Isolierter Knoten}}
\item{Grad 1 : Knoten mit Grad 1 befinden sich am Ende von Pfaden. Der angrenzende Knoten hat entweder auch
den Grad 1 (Zusammenhangskomponente mit 2 Knoten und 1 Kante ist ein \textbf{einfacher Pfad}), oder den
Grad 2}
\item{Grad 2: Es grenzt entweder ein Knoten mit Grad 1 an, oder ein weiterer Knoten mit Grad 2.}
\begin{itemize}
\item{Grad 1 : Ende eines \textbf{einfachen Pfades}. Geht man in die andere Richtung, kann es nur eine
endliche Menge von Knoten (Def. Einfacher Graph) geben, von denen der letzte den Grad 1 haben muss, sonst
wäre es nicht der letzte}
\item{Grad 2 : Kann wieder angrenzen an Knoten von Grad 1 (Ende eines \textbf{einfachen Pfades} oder wieder
an einen Knoten von Grad 2. Sind in der Zusammenhangskomponente mindestens 3 Knoten und alle vom Grad 2 so
kann es sich nur um einen \textbf{einfachen Zyklus} handeln, da es kein Ende gibt (Knoten vom Grad 1) und
jeder Knoten nur Kanten hat zur linken und rechten Seite des Zyklus} 

\end{itemize}
\end{itemize}

\end{document}

