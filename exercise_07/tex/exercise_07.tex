\documentclass{scrartcl}

\usepackage{amsmath}	  % required for math in general
\usepackage{amsthm}     % environments for theorems, qed's etc
                        % (loaded after amsmath)
\usepackage{amssymb}	  % doublestroke symbols, other mathematical symbols
\usepackage{dsfont}     % required for double-stroke 1 as characteristic function
\usepackage{array}	    % control of matrices and tables
\usepackage{graphicx}   % images

\usepackage{enumitem}   % more fine-grained control over enumerations
\setdescription{leftmargin=\parindent,labelindent=\parindent}

\usepackage{listings} % code listings
\lstset{basicstyle=\ttfamily\scriptsize}

% \input{diagrams.sty} (no category theory this time)

\usepackage{helvet}   % use (much fresher looking) helvetica for everything
\renewcommand{\familydefault}{\sfdefault}

\usepackage[weather]{ifsym}      % \Lightning symbol
% \usepackage{mathabx}             % \Asterisk causes some conflicts

% forcing the fucking floats to stop fucking floating like a fucking piece of
% shit in an ocean of fucking shit
\renewcommand{\topfraction}{.85}
\renewcommand{\bottomfraction}{.7}
\renewcommand{\textfraction}{.15}
\renewcommand{\floatpagefraction}{.66}
\renewcommand{\dbltopfraction}{.66}

% making all references into hyperlinks
\usepackage[dvipsnames]{xcolor}
\usepackage{hyperref}

\hypersetup{colorlinks=true,linkcolor=MidnightBlue,pdfborderstyle={/W 0}}

\usepackage{anyfontsize}
\usepackage{datetime}

\usepackage{fancyvrb}
% forall
\let\oldforall\forall
\renewcommand{\forall}{\oldforall\,}

% parentheses
\newcommand{\rPar}[1]{\left(#1\right)} % round parens
\newcommand{\sPar}[1]{\left[#1\right]} % square parens
\newcommand{\cPar}[1]{\left\{#1\right\}} % curved parens 
\newcommand{\aPar}[1]{\left\langle #1 \right\rangle} % angle brackets

% floor and ceiling
\newcommand{\floor}[1]{{\left\lfloor#1\right\rfloor}} % curved parens 
\newcommand{\ceil}[1]{{\left\lceil#1\right\rceil}} % curved parens 

% norms
\newcommand{\abs}[1]{\left\lvert #1\right\rvert}
\newcommand{\norm}[1]{\left\lVert #1\right\rVert}
\newcommand{\scalar}[2]{\left\langle#1,#2\right\rangle}
\newcommand{\cross}{\times}
\DeclareMathOperator{\diam}{diam}
\DeclareMathOperator{\B}{B}

% intervals
\newcommand{\openOpenInterval}[2]{{\left(#1,#2\right)}}
\newcommand{\openClosedInterval}[2]{{\left(#1,#2\right]}}
\newcommand{\closedOpenInterval}[2]{{\left[#1,#2\right)}}
\newcommand{\closedClosedInterval}[2]{{\left[#1,#2\right]}}

% restriction of functions
\newcommand{\restrict}[2]{{\left.#1\right\vert_{#2}}}

% numbers
\newcommand{\Natural}{\mathbb{N}}
\newcommand{\Integer}{\mathbb{Z}}
\newcommand{\Real}{\mathbb{R}}
\newcommand{\Rational}{\mathbb{Q}}
\newcommand{\PositiveReal}{\Real_{>0}}
\newcommand{\NonnegativeReal}{\Real_{\geq0}}
\newcommand{\Complex}{\mathbb{C}}
\renewcommand{\i}{i}
\newcommand{\Quaternion}{\mathbb{H}}
\newcommand{\Boolean}{\mathbb{B}}

% function spaces
\newcommand{\SemiLebesgue}{\mathscr{L}}
\newcommand{\Continuous}{C}
\newcommand{\Lebesgue}{L}
\newcommand{\Sobolev}{H}
\newcommand{\Hilbert}{\mathscr{H}}
\newcommand{\Schwarz}{\mathscr{S}}

% set
\newcommand{\setPredicate}[2]{{\left\{#1\,\left\vert\, #2\right.\right\}}}
\newcommand{\set}[1]{{\left\{#1\right\}}}
\newcommand{\cardinality}[1]{\left\lvert #1 \right\rvert}
\newcommand{\powerset}{\mathfrak{P}}
\DeclareMathOperator*{\intersection}{\bigcap}
\DeclareMathOperator*{\union}{\bigcup}
\newcommand{\disjointUnion}{\biguplus}
\renewcommand{\complement}[1]{#1^c}
% \newcommand{\setminus}{\backslash}
\newcommand{\injective}{\hookrightarrow}
\newcommand{\surjective}{\twoheadrightarrow}
%\DeclareMathOperator{\ker}{ker} % already exists... im does not?
\DeclareMathOperator{\im}{im}

% topological operators
\DeclareMathOperator{\Cl}{Cl}
\newcommand{\Closure}[2]{\Cl_{#1}\left(#2\right)}
\DeclareMathOperator{\const}{const}

% span and conv
\DeclareMathOperator*{\conv}{conv}
\DeclareMathOperator*{\linhull}{span}

% matrices
\newcommand{\mat}[2]{\left[\begin{array}{#1}#2\end{array}\right]}
\DeclareMathOperator*{\diag}{diag}

% landau symbols
\newcommand{\LandauO}[1]{\mathcal{O}\left(#1\right)}

% derivatives
\newcommand{\dd}[2]{\frac{\partial #1}{\partial #2}}
\newcommand{\differential}[1]{\boldsymbol{D}_{#1}}

% integrals
\renewcommand{\d}{\quad\mathrm{d}}

% characteristic functions, expected values, variances, covariances
% stochastic stuff
\newcommand{\one}[1]{\mathds{1}_{#1}}
\newcommand{\weakconv}[1]{\overset{#1}{\Longrightarrow}}
\newcommand{\wlim}{\mathop{\mathrm{wlim}}}
\newcommand{\vlim}{\mathop{\mathrm{vlim}}}
\renewcommand{\P}{\mathbb{P}}
\newcommand{\E}{\mathbb{E}}

% lim inf lim sup
% \DeclareMathOperator{\liminf}{lim inf}
% \DeclareMathOperator{\limsup}{lim sup}

% qed etc.
\renewcommand{\qedsymbol}{$\blacksquare$}
\newcommand{\result}{\hfill $\Diamond$}

% lattices
\newcommand{\meet}{\wedge}
\newcommand{\join}{\vee}
\newcommand{\negate}{\neg}

% listings: Scala
\lstdefinelanguage{scala}{
  morekeywords={abstract,case,catch,class,def,%
    do,else,extends,false,final,finally,%
    for,if,implicit,import,match,mixin,%
    new,null,object,override,package,%
    private,protected,requires,return,sealed,%
    super,this,throw,trait,true,try,%
    type,val,var,while,with,yield},
  otherkeywords={=>,<-,<\%,<:,>:,\#,@},
  sensitive=true,
  morecomment=[l]{//},
  morecomment=[n]{/*}{*/},
  morestring=[b]",
  morestring=[b]',
  morestring=[b]"""
}
\lstset{showstringspaces=false}

% making references look a little nices
\let\oldRef\ref
\renewcommand{\ref}[1]{(\oldRef{#1})}

% weird stuff for computer science
\DeclareMathOperator{\arity}{ar}

% cat, category theory
% Bunch of categories
\DeclareMathOperator{\Id}{Id}
\DeclareMathOperator{\Top}{Top}
\DeclareMathOperator{\hTop}{h-Top}
\DeclareMathOperator{\Sets}{Sets}
\DeclareMathOperator{\Rel}{Rel}
\DeclareMathOperator{\FinSets}{FinSets}
\DeclareMathOperator{\Grp}{Grp}
\DeclareMathOperator{\Cat}{Cat}
\DeclareMathOperator{\Grpd}{Grpd}
\newcommand{\cat}[1]{\mathcal{#1}}
\newcommand{\Obj}{\mathrm{Obj}}
\newcommand{\Hom}{\mathrm{Hom}}
\newcommand{\op}{\mathrm{op}}
\newcommand{\nat}{\xrightarrow{\bullet}}
\newcommand{\iso}{\cong}
\newcommand{\dom}{\mathrm{dom}}
\newcommand{\cod}{\mathrm{cod}}
\DeclareMathOperator{\coeq}{Coeq}
\newcommand{\fst}{\mathrm{fst}}
\newcommand{\snd}{\mathrm{snd}}
\DeclareMathOperator{\Aut}{Aut}
\DeclareMathOperator{\End}{End}

% functors frequently used in various contexts
\DeclareMathOperator{\Free}{Free}
\DeclareMathOperator{\Forget}{Forget}

% empty set that is round
\let\emptyset\varnothing

% generated groups
\newcommand{\gen}[1]{\left\langle#1\right\rangle}
\newcommand{\normalSub}{\triangleleft}
\newcommand{\Asterisk}{\mathop{\scalebox{1.5}{\raisebox{-0.2ex}{$\ast$}}}}
\newcommand{\Sym}{\mathrm{Sym}}

% argmax argmin argsup etc.
\DeclareMathOperator{\argsup}{argsup}
\DeclareMathOperator{\argmax}{argmax}

% number theoretic operators
\DeclareMathOperator{\lcm}{lcm}

% get rid of the ugly-looking "epsilon"
\renewcommand{\epsilon}{\varepsilon}

% get rid of the empty-looking "angle"
\renewcommand{\angle}{\measuredangle}

\newcommand{\exercise}[2]{\vspace{1em}\noindent{\bf Exercise #1 (#2)}}
\renewcommand{\proof}{\vspace{0.8em}\noindent{\bf Proof: }}

\begin{document}
\noindent{\footnotesize Game Theory 2014/15, Exercise 7} 
\hfill 
{\footnotesize \input{./currentDate.txt}}
\newline
{\footnotesize \input{../../NAMES.txt}}

\noindent\hrulefill

\exercise{7.1}{Compositions of strictly monotone functions}
Let $O$ be some set of \emph{outcomes}, 
$\preceq$ a total, reflexive and transitive relation 
(\emph{preference} relation). 
Let $u: O\times O \to \Real$ \emph{represent} the relation $\preceq$, 
that is, for all $x,y\in O$ it holds:
\[
  x \preceq y \Leftrightarrow u(x) \leq u(y).
\]
Let $v:\Real \to \Real$ be \emph{strictly} monotonously increasing. 
We claim that $v\circ u$ also represents $\preceq$.

\noindent {\bf Lemma: } If $v: \Real \to \Real$ is strictly monotonous, that
is:
\[
  a < b \Rightarrow v(a) < v(b),
\]
then it holds:
\[
  a \leq b \Leftrightarrow v(a) \leq v(b).
\]

\noindent {\bf Proof of the lemma: }
Suppose $a\leq b$. Then either $a=b$ or $a<b$. 
The first case clearly implies $v(a) = v(b)$, because $v$ is a function, not
some non-deterministic algorithm.
The second case implies by definition $v(a) < v(b)$. 
In both cases we obtain $v(a) \leq v(b)$.

Now suppose $v(a) \leq v(b)$.
If $v(a) = v(b)$, then $a=b$, because $v$ is injective.
If $v(a) < v(b)$, then $b \leq a$ cannot hold, because otherwise we would
have $v(b) \leq v(a)$. Therefore, it must hold $a < b$.
In both cases we obtain $a \leq b$.

Both directions together show the lemma. \hfill \qed

Now, the proof of the main claim that $(v\circ u)$ also represents $\preceq$ is
easy:

\noindent {\bf Proof: } For all $x,y\in O$ it holds:
\[
  x\preceq y 
    \Leftrightarrow u(x) \leq u(y) 
    \Leftrightarrow v(u(x)) \leq v(u(y)).
\]
\hfill \qed

\noindent{\bf Remark: } Injectivity of $v$ is really necessary, because 
otherwise we get only 
\[
  x\preceq y 
    \Leftrightarrow u(x) \leq u(y) 
    \Rightarrow v(u(x)) \leq v(u(y)).
\]
with $v=0$ as counterexample for the opposite implication.

\exercise{7.2}{Lex}
Define the lexicographical order on $I^2 := [0,1]^2$ as follows:
\newcommand{\lex}{\leq_{lex}}
\[
  x \lex y :\Leftrightarrow x_1 < x_y \lor (x_1 = y_2 \land x_2 \leq y_2).
\]

\noindent {\bf a)} We claim that $\leq$ is a total, reflexive and transitive
relation.

\noindent {\bf Proof: } Let $x, y\in I^2$ be two arbitrary points.
Since the usual relation $\leq$ is total on $I$, there are three possible 
cases: $x_1 < y_1$, $x_1 = y_1$ or $x_1 > y_1$. The first and last
cases are analogous, so we consider only the first two.
In the first case it holds $x \lex y$. 
In the second case, at least one of the both inequalities 
$x_2 \leq y_2$ or $y_2 \leq x_2$ must hold, therefore
it's either $x\lex y$ or $y\lex x$ respectively. 
In every case, either $x\lex y$ or $y\lex x$ holds, so $\lex$ is total.

Obviously, $x_1 = x_1$ and $x_2 \leq x_2$, therefore $x\lex x$, so $\lex$ is
reflexive. 

Now let $z\in I^2$ be a third point. 
We want to show that from $x \lex y \lex z$ it follows that $x \lex z$.
The proof can again be done by a trivial case-analysis.
However, since it's tedious and error-prone, it seems to be easier and 
safer to check all the cases by brute-force, for example using the
following little C program:
\lstinputlisting{truthTable.c}
This program generates all possible cases for all possible relations of
$x_1$, $x_2$, ..., $z_2$ and checks that $x \lex z$ holds whenever 
$x\lex y$ and $y\lex z$ hold. 
The output is a truth table with $3^6 = 729$ lines, together with a last
line that tells us whether the implication always holds. 
\lstinputlisting{truthTable.txt}
The result is $1$, which means \emph{true}, so the computer-assisted 
case analysis proves our claim. \hfill \qed 

\vspace{1.5em}

\noindent {\bf b)} We want to show that $\lex$ on $I^2$ is not
representable by any $\Real$-valued function.

Suppose for the sake of contradiction that there is some $u: I^2 \to \Real$
which represents $\lex$.
For all $x \in I$ it must hold: 
$u(x, 0) < u(x, 1)$, that is $u(x, 1) - u(x, 0)$ is positive.
Subdivide the unit interval as follows:
\[
  X_n := \setPredicate{x\in I}{u(x, 1) - u(x, 0) > \frac{1}{n}}.
\]
Since $I$ is uncountable, there must exists some $N\in \Natural$ such that
$X_N$ is also uncountable, because otherwise the unit interval $I$ would be
countable as a countable union of countable sets.
Let $M := \ceil{u(1,1)}$ be the ceiled maximum value of the function $u$,
and $m := \floor{u(0,0)}$ the floored minimum value of the function $u$.
Let $K := 2(M-m)N + 1$.
Since $X_N$ contains infinitely many elements, we can find $K$ 
distinct elements $x_1, \dots, x_K$ in $X_N$. 
Without loss of generality, we can assume that the finite sequence 
$(x_i)_{i=1}^K$ is sorted in increasing order.
It holds:
\[
  u(x_i, 1) + \frac{1}{N} < u(x_{i+1}, 0) + \frac{1}{N} < u(x_{i+1}, 1),
\]
applying this $K-1$ times to the whole sequence $(x_i)_i$ yields:
\[
  u(x_K, 1) > u(x_1, 1) + \frac{2(M-m)N}{N} = u(x_1, 1) + 2(M-m),
\]
and in particular
\[
  u(1, 1) \geq u(x_K, 1) > u(x_1, 1) + 2(M-m) \geq u(0,0) + 2(M-m).
\]
Now we obtain the situation where the difference between $u(0,0)$ and
$u(1,1)$ is at least twice as big as the difference between the 
smallest and the largest value of $u$. This is a contradiction.
Therefore there can not be a representing function for $\lex$. 
\hfill \qed

\exercise{7.3}{Harasanyi-Game}
This seems to be just a ``term-unification''/``definition-understanding'' 
task, so we just show how the concepts in the text match the symbols in
the definition of a Harasanyi-game.

The set of players is $N = \set{\textrm{buyer}, \textrm{seller}}$.
The sets of types are 
\begin{align*}
  T_{\textrm{buyer}} &= \set{ \textrm{wantCar}} \\
  T_{\textrm{seller}} &= \set{\textrm{offerGoodCar}, \textrm{offerBadCar}}.
\end{align*}
The type-dependent action sets are:
\begin{align*}
  A_{\textrm{buyer }}(\textrm{wantCar}) &= \set{\textrm{buy},\neg \textrm{buy}} \\
  A_{\textrm{seller}}(\textrm{offerGoodCar}) = A_{\textrm{seller}}(\textrm{offerBadCar}) &= \set{\textrm{sell}, \neg \textrm{sell}}.
\end{align*}
The probability distribution $\P \in \Delta(T)$ is described by a single 
constant $q\in [0, 1]$, because 
\[
  T = \set{
    (\textrm{wantCar}, \textrm{offerGoodCar}), 
    (\textrm{wantCar}, \textrm{offerBadCar})}
\]
contains just two elements, so it holds:
\begin{align*}
  \P\sPar{\set{(\textrm{wantCar}, \textrm{offerGoodCar})}} &= q \\
  \P\sPar{\set{(\textrm{wantCar}, \textrm{offerBadCar})}} &= 1 - q.
\end{align*}
The type dependent utility functions $u_1, u_2$ are given by the matrices
(we use the convention that $u_1$ and $u_2$ are written in single cell):
\begin{align*}
  u_{1,2}(\textrm{wantCar}, \textrm{offerGoodCar}) &= \mat{ccc}{
                      & \textrm{sell}   & \neg \textrm{sell} \\
    \textrm{buy}      & 6-c, c & 0, 5 \\
    \neg \textrm{buy} & 0,5    & 0, 5
  }
  \\
  u_{1,2}(\textrm{wantCar}, \textrm{offerBadCar}) &= \mat{ccc}{
             & \textrm{sell}   & \neg \textrm{sell} \\
    \textrm{buy}      & 4-c, c & 0, 0 \\
    \neg \textrm{buy} & 0,0    & 0, 0
  }.
\end{align*}
Here $c$ stands for the cost of the car.
\end{document}

                                                                                                                                                                                                                                               
                                                                                                                                                                                                                                               
                                                                                                                                                                                                                                               
                                                                                                                                                                                                                                               
                                                                                                                                                                                                                                               
                                                                                                                                                                                                                                               
                                                                                                                                                                                                                                               
                                                                                                                                                                                                                                               
                                                                                                                                                                                                                                               
