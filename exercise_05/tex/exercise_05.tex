\documentclass{scrartcl}

\usepackage{amsmath}	  % required for math in general
\usepackage{amsthm}     % environments for theorems, qed's etc
                        % (loaded after amsmath)
\usepackage{amssymb}	  % doublestroke symbols, other mathematical symbols
\usepackage{dsfont}     % required for double-stroke 1 as characteristic function
\usepackage{array}	    % control of matrices and tables
\usepackage{graphicx}   % images

\usepackage{enumitem}   % more fine-grained control over enumerations
\setdescription{leftmargin=\parindent,labelindent=\parindent}

\usepackage{listings} % code listings
\lstset{basicstyle=\ttfamily\scriptsize}

% \input{diagrams.sty} (no category theory this time)

\usepackage{helvet}   % use (much fresher looking) helvetica for everything
\renewcommand{\familydefault}{\sfdefault}

\usepackage[weather]{ifsym}      % \Lightning symbol
% \usepackage{mathabx}             % \Asterisk causes some conflicts

% forcing the fucking floats to stop fucking floating like a fucking piece of
% shit in an ocean of fucking shit
\renewcommand{\topfraction}{.85}
\renewcommand{\bottomfraction}{.7}
\renewcommand{\textfraction}{.15}
\renewcommand{\floatpagefraction}{.66}
\renewcommand{\dbltopfraction}{.66}

% making all references into hyperlinks
\usepackage[dvipsnames]{xcolor}
\usepackage{hyperref}

\hypersetup{colorlinks=true,linkcolor=MidnightBlue,pdfborderstyle={/W 0}}

\usepackage{anyfontsize}
\usepackage{datetime}

% forall
\let\oldforall\forall
\renewcommand{\forall}{\oldforall\,}

% parentheses
\newcommand{\rPar}[1]{\left(#1\right)} % round parens
\newcommand{\sPar}[1]{\left[#1\right]} % square parens
\newcommand{\cPar}[1]{\left\{#1\right\}} % curved parens 
\newcommand{\aPar}[1]{\left\langle #1 \right\rangle} % angle brackets

% floor and ceiling
\newcommand{\floor}[1]{{\left\lfloor#1\right\rfloor}} % curved parens 
\newcommand{\ceil}[1]{{\left\lceil#1\right\rceil}} % curved parens 

% norms
\newcommand{\abs}[1]{\left\lvert #1\right\rvert}
\newcommand{\norm}[1]{\left\lVert #1\right\rVert}
\newcommand{\scalar}[2]{\left\langle#1,#2\right\rangle}
\newcommand{\cross}{\times}
\DeclareMathOperator{\diam}{diam}
\DeclareMathOperator{\B}{B}

% intervals
\newcommand{\openOpenInterval}[2]{{\left(#1,#2\right)}}
\newcommand{\openClosedInterval}[2]{{\left(#1,#2\right]}}
\newcommand{\closedOpenInterval}[2]{{\left[#1,#2\right)}}
\newcommand{\closedClosedInterval}[2]{{\left[#1,#2\right]}}

% restriction of functions
\newcommand{\restrict}[2]{{\left.#1\right\vert_{#2}}}

% numbers
\newcommand{\Natural}{\mathbb{N}}
\newcommand{\Integer}{\mathbb{Z}}
\newcommand{\Real}{\mathbb{R}}
\newcommand{\Rational}{\mathbb{Q}}
\newcommand{\PositiveReal}{\Real_{>0}}
\newcommand{\NonnegativeReal}{\Real_{\geq0}}
\newcommand{\Complex}{\mathbb{C}}
\renewcommand{\i}{i}
\newcommand{\Quaternion}{\mathbb{H}}
\newcommand{\Boolean}{\mathbb{B}}

% function spaces
\newcommand{\SemiLebesgue}{\mathscr{L}}
\newcommand{\Continuous}{C}
\newcommand{\Lebesgue}{L}
\newcommand{\Sobolev}{H}
\newcommand{\Hilbert}{\mathscr{H}}
\newcommand{\Schwarz}{\mathscr{S}}

% set
\newcommand{\setPredicate}[2]{{\left\{#1\,\left\vert\, #2\right.\right\}}}
\newcommand{\set}[1]{{\left\{#1\right\}}}
\newcommand{\cardinality}[1]{\left\lvert #1 \right\rvert}
\newcommand{\powerset}{\mathfrak{P}}
\DeclareMathOperator*{\intersection}{\bigcap}
\DeclareMathOperator*{\union}{\bigcup}
\newcommand{\disjointUnion}{\biguplus}
\renewcommand{\complement}[1]{#1^c}
% \newcommand{\setminus}{\backslash}
\newcommand{\injective}{\hookrightarrow}
\newcommand{\surjective}{\twoheadrightarrow}
%\DeclareMathOperator{\ker}{ker} % already exists... im does not?
\DeclareMathOperator{\im}{im}

% topological operators
\DeclareMathOperator{\Cl}{Cl}
\newcommand{\Closure}[2]{\Cl_{#1}\left(#2\right)}
\DeclareMathOperator{\const}{const}

% span and conv
\DeclareMathOperator*{\conv}{conv}
\DeclareMathOperator*{\linhull}{span}

% matrices
\newcommand{\mat}[2]{\left[\begin{array}{#1}#2\end{array}\right]}
\DeclareMathOperator*{\diag}{diag}

% landau symbols
\newcommand{\LandauO}[1]{\mathcal{O}\left(#1\right)}

% derivatives
\newcommand{\dd}[2]{\frac{\partial #1}{\partial #2}}
\newcommand{\differential}[1]{\boldsymbol{D}_{#1}}

% integrals
\renewcommand{\d}{\quad\mathrm{d}}

% characteristic functions, expected values, variances, covariances
% stochastic stuff
\newcommand{\one}[1]{\mathds{1}_{#1}}
\newcommand{\weakconv}[1]{\overset{#1}{\Longrightarrow}}
\newcommand{\wlim}{\mathop{\mathrm{wlim}}}
\newcommand{\vlim}{\mathop{\mathrm{vlim}}}
\renewcommand{\P}{\mathbb{P}}
\newcommand{\E}{\mathbb{E}}

% lim inf lim sup
% \DeclareMathOperator{\liminf}{lim inf}
% \DeclareMathOperator{\limsup}{lim sup}

% qed etc.
\renewcommand{\qedsymbol}{$\blacksquare$}
\newcommand{\result}{\hfill $\Diamond$}

% lattices
\newcommand{\meet}{\wedge}
\newcommand{\join}{\vee}
\newcommand{\negate}{\neg}

% listings: Scala
\lstdefinelanguage{scala}{
  morekeywords={abstract,case,catch,class,def,%
    do,else,extends,false,final,finally,%
    for,if,implicit,import,match,mixin,%
    new,null,object,override,package,%
    private,protected,requires,return,sealed,%
    super,this,throw,trait,true,try,%
    type,val,var,while,with,yield},
  otherkeywords={=>,<-,<\%,<:,>:,\#,@},
  sensitive=true,
  morecomment=[l]{//},
  morecomment=[n]{/*}{*/},
  morestring=[b]",
  morestring=[b]',
  morestring=[b]"""
}
\lstset{showstringspaces=false}

% making references look a little nices
\let\oldRef\ref
\renewcommand{\ref}[1]{(\oldRef{#1})}

% weird stuff for computer science
\DeclareMathOperator{\arity}{ar}

% cat, category theory
% Bunch of categories
\DeclareMathOperator{\Id}{Id}
\DeclareMathOperator{\Top}{Top}
\DeclareMathOperator{\hTop}{h-Top}
\DeclareMathOperator{\Sets}{Sets}
\DeclareMathOperator{\Rel}{Rel}
\DeclareMathOperator{\FinSets}{FinSets}
\DeclareMathOperator{\Grp}{Grp}
\DeclareMathOperator{\Cat}{Cat}
\DeclareMathOperator{\Grpd}{Grpd}
\newcommand{\cat}[1]{\mathcal{#1}}
\newcommand{\Obj}{\mathrm{Obj}}
\newcommand{\Hom}{\mathrm{Hom}}
\newcommand{\op}{\mathrm{op}}
\newcommand{\nat}{\xrightarrow{\bullet}}
\newcommand{\iso}{\cong}
\newcommand{\dom}{\mathrm{dom}}
\newcommand{\cod}{\mathrm{cod}}
\DeclareMathOperator{\coeq}{Coeq}
\newcommand{\fst}{\mathrm{fst}}
\newcommand{\snd}{\mathrm{snd}}
\DeclareMathOperator{\Aut}{Aut}
\DeclareMathOperator{\End}{End}

% functors frequently used in various contexts
\DeclareMathOperator{\Free}{Free}
\DeclareMathOperator{\Forget}{Forget}

% empty set that is round
\let\emptyset\varnothing

% generated groups
\newcommand{\gen}[1]{\left\langle#1\right\rangle}
\newcommand{\normalSub}{\triangleleft}
\newcommand{\Asterisk}{\mathop{\scalebox{1.5}{\raisebox{-0.2ex}{$\ast$}}}}
\newcommand{\Sym}{\mathrm{Sym}}

% argmax argmin argsup etc.
\DeclareMathOperator{\argsup}{argsup}
\DeclareMathOperator{\argmax}{argmax}

% number theoretic operators
\DeclareMathOperator{\lcm}{lcm}

% get rid of the ugly-looking "epsilon"
\renewcommand{\epsilon}{\varepsilon}

% get rid of the empty-looking "angle"
\renewcommand{\angle}{\measuredangle}

\newcommand{\exercise}[2]{\vspace{1em}\noindent{\bf Exercise #1 (#2)}}
\renewcommand{\proof}{\vspace{0.8em}\noindent{\bf Proof: }}

\begin{document}
\noindent{\footnotesize Game Theory 2014/15, Exercise 5} 
\hfill 
{\footnotesize \input{./currentDate.txt}}
\newline
{\footnotesize \input{../../NAMES.txt}}

\noindent\hrulefill

\exercise{5.1}{Compactness of certain families of measures}
Let $S$ be a finite set. Consider the family of all probability measures on $S$:
\[
  \Delta(S) := \setPredicate{\sigma: S \to [0, 1]}{\sum_{s\in S}\sigma(s) = 1}.
\]
We interpret it as a subset of the finitely-dimensional space $\Real^S$, 
which can be thought of as the set of all $\Real$-vectors indexed by elements
of $S$. It is (unnaturally) isomorphic to $\Real^{\abs{S}}$ (one has to pick 
some enumeration of $S$), so all theorems that hold for the canonical 
finite-dimensional real vector space $\Real^n$ (for $n=\abs{S}$) 
also hold for $\Real^S$. In particular, the Heine-Borel theorem holds.

We want to show that $\Delta(S)$ is compact.
For all $\sigma\in\Delta(S)$ it holds:
\[
  \norm{\sigma}_{\infty} = \max_{s \in S} \abs{\sigma(s)} \leq 1,
\]
so $\Delta(S)$ is bounded.

To show that it is closed, consider the summation function:
\[
  F: \Real^S \to \Real, \qquad F(\sigma) := \sum_{s \in S}\sigma(s).
\]
We assume that it is known that $F$ is continuous. 
The single-point set $\set{1}$ is closed, so it's preimage $F^{-1}(\set{1})$ 
is closed.
The unit cube $[0,1]^S$ is closed as a finite product of closed sets.
Therefore, $\Delta(S)$ is closed as intersection of two closed sets:
\[
  \Delta(S) = [0,1]^S \cap F^{-1}(\set{1})
\]
By the Heine-Borel theorem, $\Delta(S)$ is compact.
\hfill \qed

\exercise{5.2}{Nash equilibria by Kakutani's fixed point theorem}
We want to use Kakutani's fixed point theorem\footnote{
  Notice that the statement is different from what was given in the problem
  statement. It's unclear what the canonical topology on $\powerset(X)$ should
  be.
} 
to prove the existence of Nash equilibria in strategic games with mixed 
strategies.

Let $n$ be a number of players, $S_i$ finite sets of pure strategies for each
$i=1\dots n$, $u_i: \prod_i S_i \to \Real$ utility functions, and
\[
  U_i(\sigma) \equiv U_i(\sigma_1,\dots, \sigma_n) := 
  \E[u_i(\sigma_1, \dots, \sigma_n)]
\]
the expected utilities for mixed strategies 
$\sigma_i \in \Sigma_i := \Delta(S_i)$.

\vspace{1em}

\noindent {\bf (Kakutani)}
  Let $X\subset \Real^n$ nonempty, compact and convex set.
  Let $f: X\to \powerset(X)$ such that $f(x) \neq \emptyset$ is convex for all
  $x\in X$. Let the graph of $f$
  \[
    \Gamma_f = \setPredicate{(x, y)\in X\times X}{x\in X, y\in f(x)}
  \]
  be closed in $X^2$.
  Then $f$ has a fixed point $x_0\in X$ in the sense that $x_0\in f(x_0)$.
\hfill \qed

\vspace{1em}

\noindent {\bf a)} 
Let $X := \Sigma := \prod_{i=1}^n \Sigma_i$.
As we have seen in 5.1, each $\Sigma_i$ is compact and convex. 
By $(n - 1)$-fold application of 4.2 a) we see that the product $X$ is also convex.
To see that $X$ is compact, we can either use Heine-Borel (notice that finite products of closed spaces are closed and finite products of bounded subsets of some $\Real^n_i$ are again bounded), or nuke the problem with the disproportionally general Tychonoff's theorem.

\noindent {\bf b)} Now consider the \emph{best-responses function}
\begin{align*}
  &BR_i: \Sigma \to \powerset(\Sigma_i), \quad
  BR_i(\sigma) := \argmax_{\theta\in\Sigma_i} U_i(\sigma_{-i}, \theta), \\
  &BR: \Sigma \to \powerset(\Sigma), \quad
  BR(\sigma) := \prod_{i=1}^n BR_i
\end{align*}
where $\argmax$ is interpreted as a set-function that can return multiple maxima.
We want to show that $BR_i(\sigma)$ fulfill the first condition in Kakutani's
theorem. By exercise 5.1 the set $\Sigma_i$ is compact, and $U_i$ is 
continuous, therefore the partially applied function $U_i(\sigma_{-1}, -)$ 
has at least one maximum on $\Sigma_i$ for arbitrary $\sigma$.
This shows that $BR_i(\sigma)$ is a nonempty set. 

The convexity of the sets $BR_i(\sigma)$ is a simple consequence of the 
multilinearity of the functions $U_i$. 
Let $x, y\in BR_i(\sigma)$ and $t\in[0, 1]$. 
Let $M := \max_\theta U_i(\sigma_{-1}, \theta)$.
It holds:
\[
  U_i(\sigma_{-1}, (t - 1)x + t y) = 
  (1 - t) U_i(\sigma_{-1}, x) + t U_i(\sigma_{-1}, y) = (1 - t) M + t M = M,
\]
and therefore by definition $(1-t)x + ty\in BR_i(\sigma)$. Since we already 
know from 4.2 b) that products of convex sets are again convex, we conclude 
that $BR(\sigma) = \prod_{i=1}^n BR_i(\sigma)$ is also convex (and
nonempty).

\noindent {\bf c)} Now we show that the graph $\Gamma_{BR}$ is closed.

Let $\rPar{(\sigma^n, r^n)}_n$ be a convergent sequence in $\Gamma_{BR}$.
Let $(\sigma^\ast, r^\ast) := \lim_{n\to\infty}(\sigma^n, r^n)$ be it's limit. 
We have to show that the limit also lies in $\Gamma_{BR}$.
Notice the following facts:
\begin{itemize}
  \item[(1)] All $U_i$ are continuous 
  \item[(2)] Since $\Sigma$ is compact, $U_i$ are even \emph{uniformly}
    continuous, from which it immediately follows that the function
    $\sigma \mapsto \max_{\rho\in\Sigma_i}U_i(\sigma_{-i}, \rho)$ 
    is also continuous.
  \item[(3)] Instead of considering the ill-behaved and discontinuous 
    $\argmax$ function, we can focus on the $\max$ function, since it holds:
    \[
      r_i \in BR_i(\sigma) \quad \Leftrightarrow \quad 
      U_i(\sigma_{-i}, r_i) = \max_{\rho\in\Sigma_i} U_i(\sigma_{-i}, \rho).
    \]
\end{itemize}
Using all this, we compute:
\[
  \begin{array}{ccl}
  U_i(\sigma^\ast_{-i}, r^\ast_i) 
    &\overset{\textrm{def } \sigma^\ast, r^\ast}= &
    U_i\rPar{\lim_{n\to\infty}(\sigma^n_{-i}, r^n_i)} \\
    &\overset{(1)}= &
    \lim_{n\to\infty}U_i(\sigma^n_{-i}, r^n_i) \\
    &\overset{\textrm{def } \Gamma_{BR}}= &
    \lim_{n\to\infty}\max_{\rho\in\Sigma_i}U_i(\sigma^n_{-i}, \rho) \\
    &\overset{(2)}= &
    \max_{\rho\in\Sigma_i}U_i\rPar{\lim_{n\to\infty}\sigma^n_{-i}, \rho} \\
    &\overset{\textrm{def } \sigma^\ast}= &
    \max_{\rho\in\Sigma_i}U_i(\sigma^\ast_{-i}, \rho).
  \end{array}
\]
This is by (3) equivalent to $r^\ast_i\in BR_i(\sigma^\ast)$. Since this 
holds for all $i=1,\dots,n$, we conclude that 
$(\sigma^\ast, r^\ast) \in \Gamma_{BR}$. Since this holds for all convergent
series, the graph $\Gamma_{BR}$ is closed.

\noindent {\bf d)} From the fixed point theorem of Kakutani it now follows 
that there must be a fixed point $\sigma^\dagger\in\Sigma$ such that 
\[
  \sigma^\dagger \in BR(\sigma^\dagger),
\]
which is by definition the same as to say that for all players $i$ and 
all other possible responses $r_i \in \Sigma_i$ it holds:
\[
   U_i(\sigma^\dagger) \geq U_i(\sigma^\dagger_{-i}, r_i),
\]
that is: for all players there is no profitable deviation from the strategy 
$\sigma^\dagger_i$, this is exactly the definition of a Nash-Equilibrium.
\hfill \qed

\exercise{5.2}{Evolutionary Stable States}
Consider the following symmetric game:
\[
  \mat{ccc}{
      & D & H \\
    D & (2,2) & (1,3) \\
    H & (3,1) & (7,7)
  }
\]
We claim that there is only one symmetric Nash-Equilibrium, and that it is
evolutionary stable.

Let $p$ denote the probability for the pure strategy $D$, then $(1-p)$ is the 
probability for $H$ (because of that, it is enough to specify only one 
probability, so we denote strategies simply by a single real number from $[0,1]$).
The utility for the first ``individual''-player is as follows (it's the same for the second ``population''-player):
\[
  U_i\rPar{p, p} = \mat{cc}{p & 1 - p} 
  \mat{cc}{2 & 1 \\ 3 & 7}
  \mat{c}{p \\ 1 - p}
  = 5p^2 - 10 p + 7
\]
The first derivative of this expression disappears for $p=1$, therefore
all the extrema lie on the boundary of the unit interval.
It holds (not surprisingly);
\[
  U_i(p=0, p=0) = 7 \quad U_i(p=1, p=1) = 2,
\]
therefore the only local maximum is at $p=0$, this is where the single
symmetric Nash-Equilibrium is.

Now assume that we perturb the strategy $p=0$ by an $\epsilon>0$ and
obtain a strategy $\epsilon \equiv (\epsilon, 1-\epsilon)$.
For any other strategy $q\in[0,1]$ it holds:
\[
  U_1(q, \epsilon) = 
    \mat{cc}{q, 1-q}\mat{ccc}{2 & 1 \\ 3 & 7}\mat{c}{\epsilon\\1-\epsilon} =
    5q\epsilon - 6q - 4\epsilon + 7
\]
The derivative by $d/dq$ is $5\epsilon - 6$, therefore there are no extrema
inside of the interval $[0, 1]$. On the boundary $\set{0, 1}$ it holds:
\[
  U_1(0, \epsilon) = 7 - 4\epsilon,
  U_1(1, \epsilon) = 1 - 3\epsilon,
\]
the first expression is always larger than the second, therefore the
strategy $p=0$ is always the best response to all other perturbed 
strategies $\epsilon$. This shows that $p=p_D=0$ is an evolutionary 
stable equilibrium.
\end{document}