\documentclass{scrartcl}

\usepackage{amsmath}	  % required for math in general
\usepackage{amsthm}     % environments for theorems, qed's etc
                        % (loaded after amsmath)
\usepackage{amssymb}	  % doublestroke symbols, other mathematical symbols
\usepackage{dsfont}     % required for double-stroke 1 as characteristic function
\usepackage{array}	    % control of matrices and tables
\usepackage{graphicx}   % images

\usepackage{enumitem}   % more fine-grained control over enumerations
\setdescription{leftmargin=\parindent,labelindent=\parindent}

\usepackage{listings} % code listings
\lstset{basicstyle=\ttfamily\scriptsize}

% \input{diagrams.sty} (no category theory this time)

\usepackage{helvet}   % use (much fresher looking) helvetica for everything
\renewcommand{\familydefault}{\sfdefault}

\usepackage[weather]{ifsym}      % \Lightning symbol
% \usepackage{mathabx}             % \Asterisk causes some conflicts

% forcing the fucking floats to stop fucking floating like a fucking piece of
% shit in an ocean of fucking shit
\renewcommand{\topfraction}{.85}
\renewcommand{\bottomfraction}{.7}
\renewcommand{\textfraction}{.15}
\renewcommand{\floatpagefraction}{.66}
\renewcommand{\dbltopfraction}{.66}

% making all references into hyperlinks
\usepackage[dvipsnames]{xcolor}
\usepackage{hyperref}

\hypersetup{colorlinks=true,linkcolor=MidnightBlue,pdfborderstyle={/W 0}}

\usepackage{anyfontsize}
\usepackage{datetime}

% forall
\let\oldforall\forall
\renewcommand{\forall}{\oldforall\,}

% parentheses
\newcommand{\rPar}[1]{\left(#1\right)} % round parens
\newcommand{\sPar}[1]{\left[#1\right]} % square parens
\newcommand{\cPar}[1]{\left\{#1\right\}} % curved parens 
\newcommand{\aPar}[1]{\left\langle #1 \right\rangle} % angle brackets

% floor and ceiling
\newcommand{\floor}[1]{{\left\lfloor#1\right\rfloor}} % curved parens 
\newcommand{\ceil}[1]{{\left\lceil#1\right\rceil}} % curved parens 

% norms
\newcommand{\abs}[1]{\left\lvert #1\right\rvert}
\newcommand{\norm}[1]{\left\lVert #1\right\rVert}
\newcommand{\scalar}[2]{\left\langle#1,#2\right\rangle}
\newcommand{\cross}{\times}
\DeclareMathOperator{\diam}{diam}
\DeclareMathOperator{\B}{B}

% intervals
\newcommand{\openOpenInterval}[2]{{\left(#1,#2\right)}}
\newcommand{\openClosedInterval}[2]{{\left(#1,#2\right]}}
\newcommand{\closedOpenInterval}[2]{{\left[#1,#2\right)}}
\newcommand{\closedClosedInterval}[2]{{\left[#1,#2\right]}}

% restriction of functions
\newcommand{\restrict}[2]{{\left.#1\right\vert_{#2}}}

% numbers
\newcommand{\Natural}{\mathbb{N}}
\newcommand{\Integer}{\mathbb{Z}}
\newcommand{\Real}{\mathbb{R}}
\newcommand{\Rational}{\mathbb{Q}}
\newcommand{\PositiveReal}{\Real_{>0}}
\newcommand{\NonnegativeReal}{\Real_{\geq0}}
\newcommand{\Complex}{\mathbb{C}}
\renewcommand{\i}{i}
\newcommand{\Quaternion}{\mathbb{H}}
\newcommand{\Boolean}{\mathbb{B}}

% function spaces
\newcommand{\SemiLebesgue}{\mathscr{L}}
\newcommand{\Continuous}{C}
\newcommand{\Lebesgue}{L}
\newcommand{\Sobolev}{H}
\newcommand{\Hilbert}{\mathscr{H}}
\newcommand{\Schwarz}{\mathscr{S}}

% set
\newcommand{\setPredicate}[2]{{\left\{#1\,\left\vert\, #2\right.\right\}}}
\newcommand{\set}[1]{{\left\{#1\right\}}}
\newcommand{\cardinality}[1]{\left\lvert #1 \right\rvert}
\newcommand{\powerset}{\mathfrak{P}}
\DeclareMathOperator*{\intersection}{\bigcap}
\DeclareMathOperator*{\union}{\bigcup}
\newcommand{\disjointUnion}{\biguplus}
\renewcommand{\complement}[1]{#1^c}
% \newcommand{\setminus}{\backslash}
\newcommand{\injective}{\hookrightarrow}
\newcommand{\surjective}{\twoheadrightarrow}
%\DeclareMathOperator{\ker}{ker} % already exists... im does not?
\DeclareMathOperator{\im}{im}

% topological operators
\DeclareMathOperator{\Cl}{Cl}
\newcommand{\Closure}[2]{\Cl_{#1}\left(#2\right)}
\DeclareMathOperator{\const}{const}

% span and conv
\DeclareMathOperator*{\conv}{conv}
\DeclareMathOperator*{\linhull}{span}

% matrices
\newcommand{\mat}[2]{\left[\begin{array}{#1}#2\end{array}\right]}
\DeclareMathOperator*{\diag}{diag}

% landau symbols
\newcommand{\LandauO}[1]{\mathcal{O}\left(#1\right)}

% derivatives
\newcommand{\dd}[2]{\frac{\partial #1}{\partial #2}}
\newcommand{\differential}[1]{\boldsymbol{D}_{#1}}

% integrals
\renewcommand{\d}{\quad\mathrm{d}}

% characteristic functions, expected values, variances, covariances
% stochastic stuff
\newcommand{\one}[1]{\mathds{1}_{#1}}
\newcommand{\weakconv}[1]{\overset{#1}{\Longrightarrow}}
\newcommand{\wlim}{\mathop{\mathrm{wlim}}}
\newcommand{\vlim}{\mathop{\mathrm{vlim}}}
\renewcommand{\P}{\mathbb{P}}
\newcommand{\E}{\mathbb{E}}

% lim inf lim sup
% \DeclareMathOperator{\liminf}{lim inf}
% \DeclareMathOperator{\limsup}{lim sup}

% qed etc.
\renewcommand{\qedsymbol}{$\blacksquare$}
\newcommand{\result}{\hfill $\Diamond$}

% lattices
\newcommand{\meet}{\wedge}
\newcommand{\join}{\vee}
\newcommand{\negate}{\neg}

% listings: Scala
\lstdefinelanguage{scala}{
  morekeywords={abstract,case,catch,class,def,%
    do,else,extends,false,final,finally,%
    for,if,implicit,import,match,mixin,%
    new,null,object,override,package,%
    private,protected,requires,return,sealed,%
    super,this,throw,trait,true,try,%
    type,val,var,while,with,yield},
  otherkeywords={=>,<-,<\%,<:,>:,\#,@},
  sensitive=true,
  morecomment=[l]{//},
  morecomment=[n]{/*}{*/},
  morestring=[b]",
  morestring=[b]',
  morestring=[b]"""
}
\lstset{showstringspaces=false}

% making references look a little nices
\let\oldRef\ref
\renewcommand{\ref}[1]{(\oldRef{#1})}

% weird stuff for computer science
\DeclareMathOperator{\arity}{ar}

% cat, category theory
% Bunch of categories
\DeclareMathOperator{\Id}{Id}
\DeclareMathOperator{\Top}{Top}
\DeclareMathOperator{\hTop}{h-Top}
\DeclareMathOperator{\Sets}{Sets}
\DeclareMathOperator{\Rel}{Rel}
\DeclareMathOperator{\FinSets}{FinSets}
\DeclareMathOperator{\Grp}{Grp}
\DeclareMathOperator{\Cat}{Cat}
\DeclareMathOperator{\Grpd}{Grpd}
\newcommand{\cat}[1]{\mathcal{#1}}
\newcommand{\Obj}{\mathrm{Obj}}
\newcommand{\Hom}{\mathrm{Hom}}
\newcommand{\op}{\mathrm{op}}
\newcommand{\nat}{\xrightarrow{\bullet}}
\newcommand{\iso}{\cong}
\newcommand{\dom}{\mathrm{dom}}
\newcommand{\cod}{\mathrm{cod}}
\DeclareMathOperator{\coeq}{Coeq}
\newcommand{\fst}{\mathrm{fst}}
\newcommand{\snd}{\mathrm{snd}}
\DeclareMathOperator{\Aut}{Aut}
\DeclareMathOperator{\End}{End}

% functors frequently used in various contexts
\DeclareMathOperator{\Free}{Free}
\DeclareMathOperator{\Forget}{Forget}

% empty set that is round
\let\emptyset\varnothing

% generated groups
\newcommand{\gen}[1]{\left\langle#1\right\rangle}
\newcommand{\normalSub}{\triangleleft}
\newcommand{\Asterisk}{\mathop{\scalebox{1.5}{\raisebox{-0.2ex}{$\ast$}}}}
\newcommand{\Sym}{\mathrm{Sym}}

% argmax argmin argsup etc.
\DeclareMathOperator{\argsup}{argsup}
\DeclareMathOperator{\argmax}{argmax}

% number theoretic operators
\DeclareMathOperator{\lcm}{lcm}

% get rid of the ugly-looking "epsilon"
\renewcommand{\epsilon}{\varepsilon}

% get rid of the empty-looking "angle"
\renewcommand{\angle}{\measuredangle}

\newcommand{\exercise}[2]{\vspace{1em}\noindent{\bf Exercise #1 (#2)}}
\renewcommand{\proof}{\vspace{0.8em}\noindent{\bf Proof: }}

\begin{document}
\noindent{\footnotesize Game Theory 2014/15, Exercise 9} 
\hfill 
{\footnotesize \input{./currentDate.txt}}
\newline
{\footnotesize \input{../../NAMES.txt}}

\noindent\hrulefill

\exercise{9.1}{Digital clock}
Chronos and Hora look at a digital clock 
with a seven-segment display:
\begin{verbatim}
 _       _   _       _   _   _   _   _ 
| |   |  _|  _| |_| |_  |_    | |_| |_|
|_|   | |_   _|   |  _| |_|   | |_|  _|
\end{verbatim}
Chronos sees only the four upper segments,
Hora sees only the four lower segments.
We want to investigate the states of 
knowledge at different time points 
from 00:00 to 23:59.

A philosophical remark up front. 
This task seems simple enough that one
might be tempted to try to solve it manually.
Even a randomly chosen pedestrian who has
no idea of such concepts as ``definition'' or
``logic'' (let alone ``Aumann model of incomplete information'' and ``knowledge operator''), would probably attempt to
solve it, and would possibly even come up with
an answer that resembles the correct solution.
But this answer would be based on intuition and
hand-waving, and it would not explain how to 
deal with such problems in general. 
Instead of relying on bare intuition, we shall write
a program that computes the correct answer. 
We will use Scala for scripting.

The first thing one would like to do is to generate
a description of the partitions 
that describe the events observable by Choronos and Hora.
For this, we explicitly write down the states of LEDs as
strings and perform some substring manipulations in 
order to group the numbers by their visible parts:
\begin{lstlisting}
// States of leds on a digital display
val ledStrings = Array(
  "1110111","0010010","1011101","1011011","0111010",
  "1101011","1101111","1010010","1111111","1111011"
)

// This method generates the knowledge partition 
// of the set {start, ..., end} for a player, it
// requires a function that tells us what the player
// can see. Therefore, it transforms the geometrical
// information into a simple partition of integers.
def partition(
  start: Int, end: Int, 
  visibility: String => String
) = {
  // determine number of digits (1 or 2)
  var nd = numDigits(end)

  // group numbers by the visible parts
  val p = (start to end).groupBy { k =>
    val ds = digits(nd, k)
    ds.map { digit => visibility(ledStrings(digit)) }
  }

  // `group` returns maps, 
  // we need only the right hand side
  p.values.map{_.toSet}.toSet
}
\end{lstlisting}
Here is what the string manipulations look like.
The bits are arranged in a way such that first four 
correspond to upper part, and last four correspond to
the lower part:
\begin{lstlisting}
def upperPartition(start: Int, end: Int) = 
  partition(start, end, _.take(4))
def lowerPartition(start: Int, end: Int) =
  partition(start, end, _.drop(3))
\end{lstlisting}
Now we generate two lists (one for Chronos, one for Hora) with three partitions in each list. 
The first partition is of the range $\set{0\dots 23}$, 
the second of the range $\set{0\dots 5}$ and the third 
of the range $\set{0\dots 9}$. Notice that we cannot
treat the first two digits independently, because for 
example Hora cannot tell 13 and 19 apart, but can 
recognize 23, because there are no 25 or 29.
\begin{lstlisting}
// lists of partitions for Chronos and Hora
val upperPartitions = List(23, 5, 9).map{ n => 
  upperPartition(0, n)
}

val lowerPartitions = List(23, 5, 9).map{ n =>
  lowerPartition(0, n)
}
\end{lstlisting}
This is what the partitions look like:
for Chronos:
\begin{flalign*}
\{ &\\
   & \{00\},\{01\},\{02,03\},\{04\},\{05,06\}, \\
   & \{07\},\{08,09\},\{10\},\{11\},\{12,13\}, \\
   & \{14\},\{15,16\},\{17\},\{18,19\},\{20\}, \\
   & \{21\},\{22,23\} \\
\} &\\
\{&\{0\},\{1\},\{2,3\},\{4\},\{5\}\} \\
\{&\{0\},\{1\},\{2,3\},\{4\},\{5,6\},\{ 7\},\{8,9\}\}
\end{flalign*}
for Hora:
\begin{flalign*}
\{&\\
  &\{00\},\{01,07\},\{02\},\{03,05,09\},\{04\}, \\
  &\{06,08\},\{10\},\{11,17\},\{12\},\{13,15,19\}, \\
  &\{14\},\{16,18\},\{20\},\{21\},\{22\},\{23\} \\
\}& \\
\{&\{0\},\{1\},\{2\},\{3,5\},\{4\}\} \\
\{&\{0\},\{1,7\},\{2\},\{3,5,9\},\{4\},\{6,8\}\}
\end{flalign*}
The event ``Chronos knows time exactly'' is simply the
product of unions of sets with exactly one element.
Same holds for Hora. Here are the both events:
\begin{flalign*}
  chronosKnowsTime &= 
    \{00,01,04,07,10,11,14,17,20,21\} \times 
    \{0,1,4,5\} \times
    \{0,1,4,7\} \\
  horaKnowsTime &=   
    \{00,02,04,10,12,14,20,21,22,23\} \times 
    \{0,1,2,4\} \times 
    \{0,2,4\}
\end{flalign*}

So far we did not need the knowledge operators.
One important observation about the knowledge operators is
that they respect the product structure of the involved
sets. That is, if $\mathcal{F}_i$ are partitions of 
$\Omega_i$ and $\mathcal{F} = \prod_i \mathcal{F}_i$ 
is a partition of $\Omega = \prod_i \Omega_i$, then
for the knowledge operator $K_{\mathcal{F}}$ it holds:
\begin{align*}
  K_{\mathcal{F}}(A) = \prod_i K_{\mathcal{F}_i}(A_i).
\end{align*}
This property is important for the speed of the 
computation and for a nice structure of the finite 
result: we still get a representation as a product of sets, and not a giant set of tuples.
Here is the formula expressed as code:
\begin{lstlisting}
// The knowledge operator for a single \Omega_i
def knowledge(partition: Set[Set[Int]])(a: Set[Int]) = {
  val subsetsOfA = partition.filter{x => x subsetOf a}
  subsetsOfA.flatten
}

// Knowledge operator for the whole product space
// \Omega = \prod_i \Omega_i
// implemented in terms of the previous knowledge 
// operator for a single slice
def knowledge(
  partitions: List[Set[Set[Int]]]
)(a: List[Set[Int]]): List[Set[Int]] = {
  for ((p, aComponent) <- partitions zip a) yield {
    knowledge(p)(aComponent)
  }
}
\end{lstlisting}
Now we can use the definition of the general knowledge operator
and the partitions generated previously to instantiate
the specific knowledge operators for Chronos and Hora:
\begin{lstlisting}
val k_u = knowledge(upperPartitions)(_)
val k_l = knowledge(lowerPartitions)(_)
\end{lstlisting}
Here are the results of $k_u(lowerKnowsTime)$ and
$k_l(upperKnowsTime)$:
\begin{align*}
  K_C(horaKnowsTime) &= 
    \{00,04,10,14,20,21,22,23\} \times
    \{0,1,4\} \times 
    \{0,4\} \\
  K_H(chronosKnowsTime) &=
    \{00,01,04,07,10,11,14,17,20,21\} 
    \times 
    \{0,1,4\} \times 
    \{0,1,4,7\}
\end{align*}
Notice that there are situations where for example
Hora does not know the time herself, but knows that
Chronos knows it (e.g. at 11:41).

Finally, we want to get the set of all events where
the knowledge of time becomes \emph{common knowledge}.
For this, observe that if there is a sequence of player
indices $i_1, \dots, i_n$ and for the event
\[
  C := K_{i_n}\dots K_{i_1}(A)
\]
it holds that $K_p C = C$ for all players $p$, then
$C$ is exactly the set of events where $A$ is common
knowledge. Furthermore, notice that  $K_p(A)$ is 
always contained in $A$. That means that we can 
start with an event $C_0 = A$ and then apply $K_i$ in some rather arbitrary order, until 
$K_p C = C$ for all players $p$ holds. 
Since we work with finite sets, the sequence
\[
A \supseteq K_{i_1}(A) \supseteq K_{i_2}K_{i_1}(A)
  \supseteq \dots
\]
is noetherian, so the algorithm terminates.
This is exactly what we do in the code:
\begin{lstlisting}
val ks = List(k_u, k_l)
var commonKnowledge = bothKnowTime
while (!ks.forall{k => k(commonKnowledge) == commonKnowledge}){
  println(prodToString(commonKnowledge))
  for (k <- ks) commonKnowledge = k(commonKnowledge)
}
\end{lstlisting}
This gives the set of all time-points where the time
is common knowledge:
\[
  \{00,04,10,14,20,21\} \times \{0,1,4\} x \{0,4\}
\]
See code for more details, if necessary.
\end{document}